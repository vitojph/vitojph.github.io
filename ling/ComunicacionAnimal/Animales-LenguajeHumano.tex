\documentclass[handout]{beamer}
%\documentclass{beamer}
\mode<presentation>
{ \usetheme{boxes} }
\usetheme{default}

\usepackage[spanish]{babel}
\usepackage[utf8]{inputenc}
\usepackage{tipa}
\usepackage{tipx}
\usepackage{color}
\usepackage{hyperref}

% para imprimir a dos transpas por página, cambia el documentclass y añade
\usepackage{pgfpages}
\pgfpagesuselayout{2 on 1}[a4paper,border shrink=5mm]


% items enclosed in square brackets are optional; explanation below
\title[Comunicación]{Los animales y el lenguaje humano}
%\subtitle[]{}
\author[V. Peinado]{Víctor Peinado}
\institute[UCM]{
  \texttt{v.peinado@filol.ucm.es}\\[1ex]
  
  Grado de Logopedia, Universidad Complutense de Madrid\\[1ex]
}

\date[3 oct 2011]{10 de octubre de 2011}

\begin{document}

% algunos alias
\let\it\textit
\let\ipa\textipa

%--- the titlepage frame -------------------------%
\begin{frame}[plain]
  \titlepage
\end{frame}

%--- the presentation begin here -----------------%


\begin{frame}{La comunicación animal}

\begin{itemize}
	\item Es evidente que en el reino animal existen formas de comunicación más o menos desarrolladas.
	\item Son muchos los seres vivos capaces de comunicarse (a veces, de maneras sofisticadas) con otros miembros de su propia especie.
	\item ¿Es posible encontrar alguna criatura capaz de aprender a comunicarse con los seres humanos empleando el lenguaje?
	\item ¿Posee el lenguaje humanos algunas propiedades que lo hacen especial si lo comparamos con otros mecanismos de comunicación? 
	\item Analicemos el lenguaje humanos y las distintas formas de comunicación del reino animal comparando las características de cada uno de ellos.
\end{itemize}

\end{frame}

\begin{frame}{Señales comunicativas y señales informativas}

\begin{itemize}
	\item En primer lugar, resulta preciso distinguir entre \textcolor{blue}{señales comunicativas} y \textcolor{blue}{señales informativas}. Las primeras son intencionadas, las segundas no.
	
	\item P. ej. cuando nos fijamos en una persona hablando, podemos obtener información gracias a distintas señales que esta persona nos envía de forma no intencionada.
	\begin{itemize}
		\item habla con voz ronca $\rightarrow$ está resfriada.
		\item tiene un acento diferente al mío $\rightarrow$ es extranjera.
		\item mueve continuamente un pie $\rightarrow$ es una persona nerviosa.
		\item lleva un calcetín de cada color $\rightarrow$ es algo descuidada.
	\end{itemize}
	\item Por el mimo motivo, no pensamos que un mirlo esté comunicando nada por tener plumas de determinado color, pero sí se considera que envía señales comunicativas cuando grita al advertir la presencia de un gato.
	\item La \textcolor{blue}{intencionalidad} es la principal diferencia.
\end{itemize}

\end{frame}


\begin{frame}{Desplazamiento}

\begin{itemize}
	\item Cuando llegas a casa tu gato te recibe con un \it{miau}. Lo más probable es que entendamos ese mensaje como relacionado con ese preciso instante. 
	\item Si le preguntas qué tal le ha ido el día, te responderá con un mismo \it{miau}.
	\item Parece que la comunicación animal está diseñada únicamente para el momento presente.
	\item Por el contrario, el lenguaje humano posee la propiedad de \textcolor{blue}{desplazamiento} por la que podemos hablar sobre acontecimientos pasados, futuros e incluso sobre cosas y lugares de cuya existencia no estamos seguros. 
\end{itemize}

\end{frame}

\begin{frame}{Desplazamiento}

\begin{itemize}
	\item Por regla general, se considera que la comunicación animal no puede utilizarse de forma efectiva para referirnos a sucesos situados lejos en el tiempo o el espacio.
	\item Video sobre danza de abejas: \href{http://www.youtube.com/watch?v=-7ijI-g4jHg}{http://www.youtube.com/watch?v=-7ijI-g4jHg}
	\pause
	\item En este caso, podríamos aceptar que el lenguaje de las abejas tiene una capacidad de desplazamiento limitada.
	\item Lo que parece claro es que las abejas no pueden informar sobre la situación de flores \it{en el jardín que te comenté la semana pasada} ni hablar sobre \it{el néctar que encontraremos en el paraíso de las abejas}.
\end{itemize}

\end{frame}


\begin{frame}{Arbitrariedad}

\begin{itemize}
	\item Lo normal es que no haya una conexión natural entre una forma lingüística y su significado. La conexión es \textcolor{blue}{arbitraria}.
	\item La forma lingüística no tiene una relación natural o icónica con el objeto o entidad que representa.
	\item No hay nada en la palabra \it{perro} que denote \it{mamífero ladrador de cuatro patas}. De hecho en otras lenguas se utilizan otras formas para nombrar la misma realidad: \it{dog, chien, cane, Hund}. 
	\begin{center} 
	  \includegraphics[scale=0.3]{img/palabras-iconicas.png} 
	\end{center}
	\item Como hemos visto ya, en todas las lenguas hay palabras que tratan de imitar los sonidos de la naturaleza: \it{cucú, ronrorear, tartamudear\ldots} 
	\item Esos ejemplos son escasos en todas las lenguas del mundo.
\end{itemize}

\end{frame}

\begin{frame}{Arbitrariedad}

\begin{itemize}
	\item Por el contrario, en la mayoría de las señales que emplean los animales para comunicarse sí parece existir una conexión entre el mensaje que transmiten y la señal que emplean.
	\item Da la impresión de que estas señales no son tan arbitrarias, ya que los animales cuentan con un conjunto limitado de señales.
	\item Todos los tipos de comunicación animal tienden a producirse merced a un conjunto limitado y fijo de elementos vocales o gestuales. 
	\item Estas señales se utilizan en situaciones específicas o momentos concretos: p. ej., delimitar el territorio, atraer a parejas en época de celo\ldots 
\end{itemize}

\end{frame}

\begin{frame}{Productividad}

\begin{itemize}
	\item Para poder describir nuevos objetos y situaciones, los seres humanos creamos constantemente nuevas expresiones y enunciados, a través de los recursos ligüísticos de los que disponemos.
	\item Esta propiedad se conoce con el nombre de \textcolor{blue}{productividad}, creatividad o carácter abierto.
	\item Esta productividad está directamente relacionada con el hecho de que el número potencial de enunciados de cualquier lengua humana es infitivo.
	\item Por el contrario, los sistemas de comunicación que emplean otras especies carecen de esta cualidad.
\end{itemize}

\end{frame}

\begin{frame}{Productividad}
\begin{itemize}
	\item La comunicación animal se caracteriza por tener lo que se denomina \textcolor{blue}{referencia fija}.
%	\item Las cigarras disponen de 4 señales diferentes para comunicarse con otros miembros de su especie y algunos cuentan con 36 señales vocales.
	\item Las abejas obreras disponen de un conjunto sofisticado de señales para describir distancias en el plano horizontal, pero no en el vertical $\rightarrow$ las abejas carecen de una forma de indicar \it{arriba} y ademas son incapaces de crearlo.
	\item Gritos de advertencia de cercopitecos de cara negra:
	\begin{itemize}
		\item chutter: se acerca una serpiente
		\item rraup: se acerca un águila
		\item rrr: se acerca un león
	\end{itemize}
	\item Si un buen día apareciera una serpiente voladora, los cecopitecos serían incapaces de crear una nueva señal chutter-rraup.
\end{itemize}
\end{frame}


\begin{frame}{Transmisión cultural}

\begin{itemize}
	\item Heredamos de nuestros padres determinados rasgos físicos, sin embargo no heredamos la lengua que hablamos.
	\item Adquirimos una lengua en el contexto de una cultura determinada, en relación con otros hablantes y de una forma que nada tiene que ver con los genes familiares.
	\item Un niño de origen chino que sea adoptado por padres españoles y criado en Madrid, manifestará determinadas caracterísitcas heredadas de sus padres biológicos, pero hablará castellano. 
	\item Un gatito, sometido a las mismas experiencias, dirá \it{miau} a pesar de todo.
	\item El proceso mediante el cual una lengua pasa de una generación a otra se denomina \textcolor{blue}{transmisión cultural}.
\end{itemize}

\end{frame}

\begin{frame}{Transmisión cultural}

\begin{itemize}
	\item Los humanos adquirimos nuestra lengua materna como niños en el seno de una cultura concreta.
	\item En el caso de la comunicación animal, los individuos nacen dotados de un juego de señales que se genera de forma instintiva.
	\item En determinadas especies de pájaros parece existir una combinación entre instinto y aprendizaje para producir un canto correcto. 
	\item Si estos pájaros pasan sus primeras semanas de vida sin oir a sus congéneres, producirán de manera instintiva un canto, pero serán anormales.
	\item Un bebé humano que crezca aislado no produce ninguna lengua instintiva.
	\item La transmisión cultural es crucial en el proceso de adquisición característico de las lenguas humanas.
\end{itemize}

\end{frame}

\begin{frame}{Dualidad}

\begin{itemize}
	\item El lenguaje humano está organizado simultáneamente en dos niveles o capas. Esta propiedad se conoce con el nombre de \textcolor{blue}{dualidad} o doble articulación.
	\item En la producción del habla, podemos, desde un punto de vista físico, generar sonidos individuales [e] [r] [s].
	\item Ninguna de estas formas discretas posee un significado por sí misma. 
	\item Sin embargo, cuando los emitimos de manera combinada y siguiendo un orden concreto, como en \ipa{[ser]} nos encontramos en un nivel diferente, dado que el significado de esa combinación difiere del de \ipa{[res]}.
	\item En el primer nivel tenemos sonidos diferentes, en el segundo tenemos significados diferentes.
	\item Esta dualidad de niveles permite al lenguaje humano funcionar como un sistema económico: \textcolor{blue}{a partir de un conjunto finito de unidades fónicas podemos generar un número extremadamente elevado de combinaciones con significados distintos}.
\end{itemize}

\end{frame}

\begin{frame}{Dualidad}

\begin{itemize}
	\item Esta caracterísitca no la encontramos en el lenguaje animal.
	\item Cada señal comunicativa parece consistir en una únida forma fijada, que no puede descomponerse.
	\item El \it{guau} que emite tu perro cuando llegas a casa no puede separarse en unidades más pequeñas que se puedan combinar para generar otras unidades.
	\item El \it{chutter} y el \it{rraup} de los cercopitecos tampoco se puede descomponer .
\end{itemize}


\end{frame}

\begin{frame}{Características de las lenguas naturales (Hockett, 1960)}

\begin{enumerate}
	\item Canal vocal-auditivo: las señales son emisiones vocales y son recibidas por vía auditiva.
	\item Transmisión difundida y recepción dirigida: la señal es una onda que se expande en todas direcciones; el receptor puede localizar al emisor por la dirección de la onda.
	\item Transitoriedad: las señales sonoras se desvanecen rápidamente.
	\item Desarrollo interlocutivo: un interlocutor puede tanto emitir como recibir mensajes.
	\item Retroalimentación: un emisor puede escucharse a sí mismo en el momento de emitir un mensaje.
	\item Especialización: los órganos que intervienen en la producción del hablar están especializados. 
	\item Semanticidad: las señales son realidades perceptibles sensorialmente y además transmiten significado $\rightarrow$ significante vs significado
\end{enumerate}

\end{frame}

\begin{frame}{Características de las lenguas naturales (Hockett, 1960)}

\begin{enumerate}
	\item[8.] Arbitrariedad o convencionalidad: las señales son independientes del objeto que designan.
	\item[9.] Discreticidad: las unidades básicas son separables. 
	\item[10.] Desplazamiento: es posible hacer referencia a acontecimientos pasados y futuros.
	\item[11.] Dualidad o composicionalidad: dos niveles estructurales $\rightarrow$ nivel fonico vs nivel léxico.
	\item[12.] Productividad: el número de mensajes es infinito.
	\item[13.] Transmisión cultural: el lenguaje humano es producto de la evolución histórica y se transmite entre generaciones.  
	\item[14.] Prevaricación: disimulación o falsificación, metáfora, ironía, mentira.
	\item[15.] Reflexibilidad: las lenguas se pueden utilizar para hablar de las lenguas mismas $\rightarrow$ función metalingüística.
\end{enumerate}

\end{frame}



	
\begin{frame}{Referencias}

	\begin{itemize}
		\item \href{http://es.wikipedia.org/wiki/Lengua\_natural\#Caracter.C3.ADsticas\_de\_las\_lenguas\_naturales}{Las características del lenguaje, según Charles Hockett}
		\item Bernárdez, E. \it{¿Qué son las lenguas?} Alianza Ensayo. 2004.
		\item Tusón Valls, J. \it{Introducción al lenguaje}. UOC. 2003.
		\item Yule, G. \it{El lenguaje}. Ediciones AKAI. 2007. 
	\end{itemize}

\end{frame}

\end{document}

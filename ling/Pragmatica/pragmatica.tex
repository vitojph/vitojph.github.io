\documentclass{beamer}
%\documentclass[handout]{beamer}
\usetheme{default}
\usepackage[spanish]{babel}
\usepackage[utf8]{inputenc}
%\usepackage{qtree}
\usepackage{amsmath}
\usepackage{textcomp}
%\usepackage{tipa}
%\usepackage{multirow}

%\usepackage{pgfpages}
%\pgfpagesuselayout{2 on 1}[a4paper,border shrink=5mm]


% items enclosed in square brackets are optional; explanation below
\title[]{Pragmática}
%\subtitle[Signo]{El Signo}
\author[V. Peinado]{Víctor Peinado}
\institute[UCM]{
  \texttt{v.peinado@filol.ucm.es}\\[1ex]
  
  Grado de Logopedia, Universidad Complutense de Madrid\\[1ex]
}

\date[Febrero 2011]{28 de abril - 12 de mayo de 2011}

\begin{document}

\let\ipa\textipa
\let\it\textit


%--- the titlepage frame -------------------------%
\begin{frame}[plain]
  \titlepage
\end{frame}

%--- the presentation begin here -----------------%

\begin{frame}[Parte 1]

\begin{center}
  \LARGE{Introducción a la Pragmática}
\end{center} 
\end{frame}

\begin{frame}{La pragmática}

\begin{itemize}
	\item Usamos el lenguaje todos los días, pero rara vez nos paramos a pensar en los mecanismos ocultos que hacen funcionar la comunicación.
	\item Las palabras significan por sí mismas y, sin embargo, la comunicación exige mucho más que intercambiar significados preestablecidos.
	\item \it{¿Qué quiere decir esta palabra?} $\rightarrow$ pedimos información sobre el significado que aparece en el diccionario.
	\item \it{¿Qué quieres decir con esta palabra?} $\rightarrow$ planteamos un problema de interpretación relacionado con la intención del hablante.
\end{itemize}

\end{frame}


\begin{frame}{La pragmática}

\begin{itemize}
	\item La \textcolor{blue}{Pragmática} se ocupa de esta segunda dimensión del significado: analiza el lenguaje en uso, los procesos por medio de los cuales producimos e interpretamos significados cuando usamos el lenguaje.
	\item El estudio del lenguaje tiene más de 2000 años de historia. Sin embargo, la Pragmática es el primer intento de hacer, dentro de la Lingüística, una teoría del significado de las palabras en su relación con hablantes y contextos.
	\item Para estudiar fenómenos como la interpretación de un enunciado y la relación entre el significado literal y el significado comunicado, es preciso volverse hacia el hablante y observar cómo utiliza el lenguaje y qué hace con él.
	\item Somos nosotros los que nos comunicamos, no nuestros mensajes.
\end{itemize} 

\end{frame}


\begin{frame}{El significado del hablante}

\begin{itemize}
	\item La pragmática se ocupa de estudiar el significado de las palabras, oraciones o fragmentos de oraciones pero no de manera aislada, sino usadas en actos de comunicación.
	\item \it{Qué quieres decir con\ldots?}
	\item El significado del lenguaje usado en un acto de comunicación suele conocerse con el nombre de \textcolor{blue}{significado del hablante} y se caracteriza por:
		\begin{itemize}
			\item ser intencional.
			\item depender de las circunstancias en que se produce el acto de la palabra.
		\end{itemize}
		\item Los fenómenos lingüísticos de los que se ocupa típicamente la Pragmática se ilustran en tres problemas siguientes: el significado no convencional, las relacines entre sintaxis y contexto, y la deixis.
\end{itemize}

\end{frame}


\begin{frame}{El significado no convencional}

\begin{itemize}
	\item Solemos dar por sentado que las lenguas naturales funcionan como códigos: sistemas que emparejan signos y mensajes de manera constante.
	\item La lengua establece una relación diádica, convencional y arbitraria entre representaciones fonológicas (significantes) y representaciones semánticas (significados).
	\item Habitualmente, partimos de la base de que, cuando nos comunicamos, simplemente codificamos y descodificamos información: elegimos las representaciones fonológicas que se corresponden con el contenido semántico que deseamos transmitir.
\end{itemize}

\end{frame}


\begin{frame}{El significado no convencional}

(1) Voltaire

\vspace{0.5cm}

Cuando un diplomático dice \it{sí}, quiere decir `quizá';

cuando dice \it{quizá}, quiere decir `no';

y cuando dice \it{no} no es un diplomático.

Cuando una dama dice \it{no}, quiere decir `quizá';

cuando dice \it{quizá}, quiere decir `sí';

y cuando dice \it{sí} no es una dama.


\end{frame}

\begin{frame}{El significado no convencional}

(2) Lewis Carroll

\vspace{0.5cm}

- Aquí tienes una gloria. 

- No sé que quiere usted decir con \it{una gloria} ---dijo Alicia.

- Por supuesto que no lo sabes\ldots a menos que yo te lo diga. He querido decir `aquí tienes un argumento bien apabullante'.

- ¡Pero \it{gloria} no significa `argumento bien apabullante'!

- Cuando yo uso una palabra, esa palabra significa exactamente lo que yo decido que signifique\ldots ni más ni menos. 

- La cuestión es si uno puede hacer que las palabras signifiquen cosas tan diferentes.

- La cuestión es, simplemente, quién manda aquí. 

\end{frame}

\begin{frame}{El significado no convencional}

\begin{itemize}
	\item (1) y (2) parecen un ataque frontal a la consideración de la lengua como código, ya que afirma que las palabras pueden tener un valor diferente al que le asigna el sistema.
	\item (1) es aceptable. A partir de nuestro conocimiento del mundo, podemos imaginar el tipo de peticiones que recibían los diplomáticos y las damas de la época de Voltaire, el tipo de interlocutores, etc. Contextualizamos e inferimos que ni a los diplomáticos ni a las damas les convenía hablar abiertamente.
	\item (2) resulta inaceptable y absurdo, no hay ningún tipo de inferencias que podamos realizar para poder explicar la arbitrariedad que comete Humpty-Dumpy. 
		\item No podemos explicar determinados significados si acudir a conceptos como \it{interlocutor, situación, contexto\ldots} que quedan fuera del sistema lingüístico.
\end{itemize}

\end{frame}

\begin{frame}{Sintaxis y contexto}

\begin{itemize}
	\item Sabemos que hay lenguas que presentan un orden de palabras relativamente libre y otras en el que es más bien fijo.
	\item Las lenguas tienen, básicamente, dos procedimientos para establecer las relaciones de dependencias de los constituyentes: el oden de palabras y la morfología.
	\item Se puede observar la siguiente correlación: cuánto mejor caracterizadas desde el pdv morfológico estén las relaciones sintácticas, menor necesidad habrá de marcarlas con el orden de palabras.
	\item El latín suele citarse como ejemplo de lengua con orden de palabras libre, por su riqueza morfológica. El inglés suele citarse como ejemplo opuesto: una lengua con orden de palabras fijo y muy marcado. 
\end{itemize}

\end{frame}


\begin{frame}{Sintaxis y contexto}

\begin{itemize}
	\item Latín (3)
	\begin{itemize}
		\item Caius amat Liuiam. (\it{Cayo ama a Livia}).
		\item Caius Liuiam amat.
		\item Amat Caius Liuiam.
		\item Amat Liuiam Caius.
		\item Liuiam amat Caius.
		\item Liuiam Caius amat.
	\end{itemize}
	
	\item Inglés (4)
	\begin{itemize}
		\item John loves Mary. (\it{Juan ama a María}).
		\item * John Mary loves.
		\item * Loves John Mary.
		\item * Loves Mary John.
		\item \# Mary loves John. (\it{María ama a Juan})
		\item * Mary John loves.
	\end{itemize}

\end{itemize}

\end{frame}

\begin{frame}{Sintaxis y contexto}

	Castellano (5)

	\vspace{0.5cm}
	
	\begin{itemize}
		\item Juan ama a María.
		\item A María la ama Juan.
		\item Juan a María la ama.
		
		\pause
		
		\item Las tres oraciones son equivalentes dado que describen el mismo estado de cosas. No se puede decir que una sea verdadera y las otras falsas sin caer en una grave contradicción.
		\item Si fuesen totalmente equivalentes, deberían poder intercambiarse en todos los contextos; pero esto no es así.
	\end{itemize}


\end{frame}


\begin{frame}{Sintaxis y contexto}

	Castellano (5)

	\vspace{0.5cm}
		
	\begin{itemize}
		\item Juan ama a María,
		\begin{itemize}
			\item no la odia.
			\item * no Pedro.
			\item no a Rosa.
		\end{itemize}
		\item A María la ama Juan,
		\begin{itemize}
			\item * no la odia.
			\item no Pedro.
			\item * no a Rosa.
		\end{itemize}
		\item Juan a María la ama,
		\begin{itemize}
			\item no la odia.
			\item * no Pedro.
			\item * no a Rosa.
		\end{itemize}
	\end{itemize}

\end{frame}

\begin{frame}{Sintaxis y contexto}

	\begin{itemize}
		\item Cada uno de los diversos órdenes de palabras trata a cada constituyente oracional de forma diferente desde el pdv comunicativo.
		\item En cada oración existe una parte de contenido informativo que se presenta como conocimiento compartido por los interlocutores y otra parte que se presenta como información nueva.
		\item Esta no es una peculiaridad de castellano. Otras lenguas con orden de palabras libre presentan efectos similares.
		\item En inglés, p. ej. el centro de atención dentro de la frase se marca con distinta prominencia en la pronunicación.
		\begin{itemize}
			\item John loves MARY, (not ANNA).
			\item John LOVES Mary, (he DOESN'T HATE her).
			\item JOHN loves Mary, (PETER DOESN'T).
		\end{itemize}
	\end{itemize}

\end{frame}

\begin{frame}{Sintaxis y contexto}

	\begin{itemize}
		\item A la vista de todo lo anterior, parece claro que solo podemos hablar de un orden de palabras libre si se adopta una perspectiva formal. 
		\item P. ej. en español no hay ninguna regla sintáctica que impida ninguno de los órdenes vistos en (5), sin embargo, el empleo de cada variante está estrictamente condicionado por el conocimiento previo de la situación.
		\item En resumen, si contemplamos los hechos desde un pdv general, parece claro que la algunos aspectos típicamente gramaticales (como el orden de las palabras) están determinados por factores contextuales.
		\item No podemos explicar determinados contrastes si acudir a conceptos como \it{interlocutor, situación, contexto, conocimiento compartido\ldots}
	\end{itemize}

\end{frame}


\begin{frame}{Referencia y deixis}

\begin{itemize}
	\item Desde el pdv de la comunicación, comprender una frase no consiste simplemente en recuperar significados.
	\item Es preciso también identificar referentes: no basta con entender palabras, sino saber a qué objetos, hechos o situaciones se refieren.
	\item La asignación de referencias es un paso previo a la comprensión de la frase.

	\it{Si no cierras la puerta, Kiko se escapará.}
	
	\item Otros ejemplos, cargados de deícticos, nos obligan a admitir que no sabemos a qué se refiere un mensaje ni podemos identificar las implicaciones si no conocemos el contexto.

	Una nota de papel con el mensaje: \it{Te espero mañana donde siempre.} 
	
	\item Existe un gran número de enunciados que solo podemos interpretar desde una perspectiva pragmática. 
\end{itemize}
\end{frame}


\begin{frame}{Conceptos básicos del análisis pragmático}

El análisis pragmático que proponemos está constituido por dos clases de elementos:

\begin{itemize}
	\item Componentes materiales: elementos de naturaleza física, entidades objetivas y descriptibles $\rightarrow$ emisor, destinatario, enunciado y entorno.
	\item Componentes relacionales: elementos de naturaleza inmaterial, las relaciones que se establecen entre los primeros $\rightarrow$ la información pragmática, la intención y la relación social. 
\end{itemize}

\end{frame}

\begin{frame}{Componentes materiales: el emisor}

\begin{itemize}
	\item Con el nombre \textcolor{blue}{emisor} se designa a la persona que produce intencionalmente una expresión lingüística en un momento dado.
	\item El término proviene de la teoría de la información, aunque el concepto en Pragmática adquiere precisiones ligeramente diferentes.
	\item No nos referimos a un mero codificador o transmisor de información, sino a un sujeto real, con sus conocimientos, creencias y actitudes, capaz de establecer una red de relaciones con su entorno.
	\item Está relacionado también con la idea de hablante, utilizado tradicionalmente en gramática, pero con matices. 
	\item La condición de hablante es de carácter abstracto y no se pierde nunca. El emisor es un hablante que hace uso de la palabra en un determinado momento y solo mientras emite su mensaje. 
\end{itemize}

\end{frame}


\begin{frame}{Componentes materiales: el destinatario}

\begin{itemize}
	\item Con el nombre \textcolor{blue}{destinatario} se designa a la persona (o personas) a las que el emisor dirige su enunciado.
	\item Frente a receptor, el emisor se refiere solo a sujetos, y no a simples mecanismos de descodificación.
	\item Destinatario se opone a oyente de la misma forma que emisor contrasta con hablante.
	\item Un oyente es todo aquel que tienen la capacidad abstracta de comprender un determinado código lingüístico. Destinatario es la persona a la que se dirige el mensaje.
	\item La intencionalidad es una característica distintiva. No se puede considerar destinatario a un receptor cualquiera o a un oyente ocasional.
	\item El destinatario es siempre el receptor elegido por el emisor. Y además, le mensaje está construido específicamente para él.	 
\end{itemize}

\end{frame}


\begin{frame}{Componentes materiales: el enunciado}

\begin{itemize}
	\item Llamamos \textcolor{blue}{enunciado} a la expresión lingüística que produce el emisor.
	\item Desde el pdv físico, no es más que un estímulo: auditivo o visual.
	\item Frente a la noción más general de mensaje, que puede designar cualquier tipo de información transmitida utilizando un código, el enunciado se refiere a un mensaje específicamente construido según un código lingüístico.
	\item Los límites del enunciado están fijados por la propia dinámica del discurso. Un enunciado es cada una de las intervenciones de un emisor, enmarcadas entre dos pausas. 
\end{itemize}

\end{frame}

\begin{frame}{Oración-Enunciado}

Oración (unidad de la gramática):

\begin{itemize}
	\item entidad abstracta, teórica, no realizada.
	\item se define dentro de una categoría gramatical.
	\item su contenido semántico depende de sus estructrura, no de sus usos.
	\item se evalúa en términos formales: es correcta o incorrecta.
\end{itemize}

Enunciado (unidad del discurso):

\begin{itemize}
	\item secuencia lingüística concreta realizada por un emisor en una situación comunicativa.
	\item se define dentro de una teoría pragmática.
	\item su interpretación depende de su contenido semántico y de sus condiciones de emisión.
	\item se evalúa según criterios pragmáticos: es adecuado o inadecuado.
\end{itemize}


\end{frame}


\begin{frame}{Componentes materiales: el entorno}

\begin{itemize}
	\item El cuarto elemento en el que se configura materialmente el acto comunicativo es el \textcolor{blue}{entorno}: el soporte físico, el decorado en el que se realiza la enunciación $\rightarrow$ tiempo y lugar.
	\item Coseriu distingue dentro de un «contexto extraverbal» todas las circunstancias no lingüísticas que se perciben directamente o que son conocidas por el hablante: contexto físico, contexto empírico, contexto natural, contexto práctico, contexto histórico y contexto cultural.
	\item Todos estos elementos contribuyen de manera decisiva en la comunicación, pero solo el contexto físico es un factor externo, material y descriptible objetivamente.
\end{itemize}

\end{frame}

\begin{frame}{Componentes relacionales: la información pragmática}

\begin{itemize}
	\item Por \textcolor{blue}{información pragmática} entendemos un conjunto de conocimientos, creencias, supuestos, opiniones y sentimientos de un individuo en un momento cualquiera de la interacción verbal.
	\item Emisor y destinatario, en cuanto sujetos, poseen una serie de experiencias anteriores relativas al mundo, a lo que les rodea y a los demás: interiorización de la realidad objetiva unida a las manías más personales. 
	\item La información pragmática es de naturaleza subjetiva, pero eso no implica que los interlocutores no compartan información. De hecho, los interlocutores solemos compartir conocimientos científicos, opiniones estereotipadas, o visiones del mundo comunes a nuestra cultura.
	\item Además de esa parcela común, la información pragmática de cada uno contiene una «teoría» sobre la información pragmática del otro y sobre lo que se comparte $\rightarrow$ hipótesis del conocimiento mutuo. 
\end{itemize}

\end{frame}

\begin{frame}{Componentes relacionales: la intención}

\begin{itemize}
	\item Definimos la \textcolor{blue}{intención} como la relación que existe, de un lado, entre el emisor y su información pragmática, y de otro, el destinatario y el entorno. 
%	\item Se manifiesta siempre como una relación dinámica, de voluntad de cambio. 
	\item Toda actividad humana consciente y voluntaria se concibe como el reflejo de una actitud ante su entorno. Parece legítimo preguntarse por cuál es la intencionalidad de los actos y decisiones.
	\item El mismo acto de romper el silencio y hablar indica determinada intención. 
	\item El emisor pretende actuar de determinada manera sobre el estado de las cosas preexistente, ya sea para modificarlo, ya sea para impedir que se lleve a cabo una modificación previsible.
	\item Desde la perspectiva del destinatario, el reconocimiento de la intención del emisor es un paso ineludible para la correcta interpretación de los enunciados. 
\end{itemize}

\end{frame}


\begin{frame}{Componentes relacionales: la relación social}

\begin{itemize}
	\item El tercer tipo de relación que tendremos en cuenta es la \textcolor{blue}{relación social} que existe entre los interlocutores por el mero hecho de pertenecer a una organización humana con estructura social.
	\item El papel de esta relación es fundamental porque, como habíamos dicho, el emisor construye el enunciado a la medida del destinatario.
	\item La relación social impone una serie de selecciones que determinan la forma final del enunciado.
	\item ¿Entendemos la cortesía como un conjunto de normas sociales que regulan el comportamiento de sus miembros o como una estrategia conversacional destinada a mitigar los conflictos entre mi objetivo y el de mis interlocutores?
\end{itemize}

\end{frame}


\begin{frame}{}
\begin{center}
  \LARGE{Austin y la filosofía del lenguaje corriente}
\end{center} 
\end{frame}

\begin{frame}{J. L. Austin}

\begin{itemize}
	\item J. L. Austin es un filósofo británico del s. XX, profesor de la U. Oxford. 
	\item Inició una de las líneas de investigación pragmáticas más importantes del pensamiento contemporáneo.
	\item Aunque Austin no habla explícitamente de Pragmática (era filósofo del lenguaje), sí podemos enmarcar sus aportanciones en lo que hoy consideramos como campo de esta disciplina.
	\item Fue uno de los primeros autores en estudiar la lengua corriente de cada día, frente a los análisis previos que solo consideraban válidos los lenguajes científicos y filosóficos.
	\item El lenguaje que utilizamos en nuestro comunicación ordinaria es una herramienta pulida y útil, perfectamente adaptada a los fines a los que sirve. 
	\item Su obra más conocida es una colección de conferencias titulada \it{How to Do Things with Words} (Cómo hacer cosas con palabras.
\end{itemize}

\end{frame}

\begin{frame}{Contra el verificacionalismo y la falacia descriptiva}

\begin{itemize}
	\item Buena parte de la lógica y de la filosofía del lenguaje está edificada sobre la idea de que las proposiciones son verdaderas o falsas, de acuerdo con su correspondencia o no con la realidad.
	\item Sin embargo, existen enunciados que no pueden ser juzgados como verdaderos o falsos:
	
	\it{¿Cuándo has llegado?}

	\it{¡Sal de la habitación inmediatamente!}

	\it{Ojalá dejara de llover\ldots}
	
	\item Y otras muchas situaciones, nos encontramos con un determinado enunciado que no es tanto verdadero o falso como adecuado o inadecuado.  
	\item Es más, si analizamos los diferentes usos que le damos al lenguaje, podemos deducir que su uso no es exclusivamente descriptivo. 
\end{itemize}

\end{frame}


\begin{frame}{Distinción entre oración y enunciado}

\begin{itemize}
	\item Austin añadió una precisión fundamental: ser verdadero o falso no es una propiedad instrínseca de las oraciones, sino de los enunciados. 
	\item oración $\rightarrow$ una estructura gramatical, abstracta, no realizada.
	\item enunciado $\rightarrow$ realización concreta de una oración, emitida por un hablante concreto en unas circunstancias determinadas.
	\item Una misma oración puede formar distintos enunciados emitidos por diferentes emisores en momentos distintos.
	\item La verdad o falsedad dependerá de la situación extralingüística, no de la estructura de la oración.
\end{itemize}

\end{frame}

\begin{frame}{Palabras y acciones: enunciados realizativos}

\begin{itemize}
	\item Por un lado, existe un tipo de \textcolor{blue}{enunciados constatativos} que describen el estado de las cosas y pueden evaluarse en términos de verdadero o falso.

	\it{Está lloviendo.}

	\it{Juan es alto.}

	\item Por otro lado, existe un tipo de \textcolor{blue}{enunciados realizativos o performativos} que presentan las siguientes características típicas:
	\begin{itemize}
		\item desde el pdv gramatical, se trata de una oración declarativa.
		\item va en 1ª pers. singular del presente de indicativo.
		\item no se trata de una expresión carente de sentido, pero;
		\item no puede calificarse como verdadera o falsa, sino como adecuada o inadecuada.
	\end{itemize}
	
	\it{Te pido disculpas.}

	\it{Bautizo este barco con el nombre de Lolita.}

	\it{Le apuesto 10 euros a que mañana lloverá.}
\end{itemize}

\end{frame}

\begin{frame}{Palabras y acciones: enunciados realizativos}

\begin{itemize}
	\item Al emitir cualquiera de los enunciados realizativos anteriores, el hablante no está meramente registrando el estado de las cosas, transmitiendo información o describiendo una acción $\rightarrow$ la está haciendo.
	\item Los enunciados realizativos se utilizan para llevar a cabo diferentes tipos de acciones, más o menos ritualizadas, cuyo episodio principal consiste precisamente en pronunciar determinadas palabras.
	\item Es su carácter de acción (y no de descripción) lo que confiere a los enunciados realizativos sus propiedades especiales: aparecen en 1ª pers. sing. del presente y cuando no aparecen, no tienen lectura realizativa.
	\item Poseen condiciones especiales de emisión, «las palabras tienen que decirse por las personas adecuadas en circunstancias apropiadas». En caso contrario, el acto falla. 
\end{itemize}

\end{frame}


\begin{frame}{Los infortunios}

\begin{itemize}
	\item La idea de que los enunciados realizativos puedan ser inadecuados o desafortunados lleva a Austin a desarrollar la teoría de los infortunios.
	\item La emisión que la emisión de determinadas palabras es, en muchos casos, un paso obligado en la realización de determinadas acciones de tipo convencional.
	\item No basta con las palabras: éstas tienen que emitirse siempre en las condiciones requeridas.
	\item Un fallo o violación en algunas de estas condiciones da lugar a un \textcolor{blue}{infortunio}.
\end{itemize}

\end{frame}

\begin{frame}{Los infortunios}

\begin{itemize}
	\item A1. Existencia de un procedimiento convencional que incluye la emisión de determinadas palabras y en determinadas circunstancias; además,
	\item A2. las personas y circunstancias que concurren deben ser las apropiadas para el procedimiento.
	\item B1. Todos los participantes deben actuar de la forma requerida por el procedimiento; además,
	\item B2. deben hacerlo así en todos los pasos necesarios.
	\item $\Gamma$1. Cuando el procedimiento requiere que las personas que lo realizan alberguen ciertos pensamientos, deben tenerlos; además
	\item $\Gamma$2. los participantes deben comportarse efectivamente de acuerdo con tales pensamientos.
\end{itemize}

\end{frame}

\begin{frame}{Los infortunios}

\begin{itemize}
	\item Todos los actos convencionales se prestan a sufrir infortunios, y la naturaleza y las consecuencias de éste dependerán de las condiciones que no se cumplan. 
	\item Si el fallo está en las condiciones A (si no existe el procedimiento), el infortunio se denomina \textcolor{blue}{mala apelación al procedimiento}.
	\item Si el fallo atañe a B (los pasos requeridos se llevan a cabo incorrectamente) hablamos de \textcolor{blue}{mala ejecución del procedimiento}.
	\item En ambos caso, la consecuencia es que el acto realizado es nulo o carente de efecto y Austin los denomina \textcolor{blue}{desaciertos}.
	\item La violación de las reglas $\Gamma$ provoca actos válidos en su forma externa pero \textcolor{blue}{huecos}, sin el contenido necesario. Austin los denomina \textcolor{blue}{abusos}.
\end{itemize}

\end{frame}

\begin{frame}{Los infortunios}

\begin{itemize}
	\item Desaciertos (A + B) $\rightarrow$ provocan actos nulos
	\begin{itemize}
		\item A: malas apelaciones
		\begin{itemize}
			\item A1. ?
			\item A2. malas aplicaciones
		\end{itemize}
		\item B: malas ejecuciones
		\begin{itemize}
			\item B1. actos viciados
			\item B2. actos inconclusos
		\end{itemize}
	\end{itemize}
	\item Abusos ($\Gamma$) $\rightarrow$ provocan actos huecos
	\begin{itemize}
		\item $\Gamma$1. actos insinceros
		\item $\Gamma$2. incumplimientos
	\end{itemize}
\end{itemize}

\end{frame}

\begin{frame}{Realizativos vs. Constatativos}

\begin{itemize}
	\item Cuando Austin terminó de caracterizar los enunciados se dio cuenta de que había excepciones y encontró numerosos enunciados realizativos que no aparecían, p. ej. en 1ª pers. sing. de presente indicativo.
	
	\it{Se advierte a los pasajeros que\ldots}
	
	\it{La compañía advierte a los pasajeros que\ldots}

	\it{Por la presente, está usted autorizado a\ldots}

	\it{Por la presente, el Ministerio le concede la autorización que usted solicitó para\ldots}
	
	\item Por otro lado, encontró ejemplos expresiones en 1ª pers. sing. presente que no daban lugar a enunciados realizativos.
	
	\it{Yo solo prometo algo cuando sé que puedo cumplirlo.}

	\it{Todas las semanas apuesto 20 euros a las carreras.}

	\item No todos los actos tienen su verbo realizativo correspondiente.
	
	Para disculparse es necesario decir algo parecido a «Me disculpo». Para insultar, no es necesario avisar con «te insulto».
\end{itemize}

\end{frame}

\begin{frame}{Locutivo, ilocutivo, perlocutivo}

\begin{itemize}
	\item La distinción entre estos tres tipos de actos es teórica, dado que se realizan a la vez y simultáneamente.

	\item El \textcolor{blue}{acto locutivo} es el que realizamos por el mero hecho de «decir algo». En consecuencia, puede definirse como «la emisión de ciertos ruidos, de ciertas palabras en una determinada construcción y con un cierto significado» $\rightarrow$ es el acto por el que se produce el significado.
	
	\it{Me dijo: «Dáselo a ella».}
	
	\item El \textcolor{blue}{acto ilocutivo} es el que se realiza al decir algo (\it{in saying something}) $\rightarrow$ es el acto que posee la fuerza, es el poder de hacer. 

	\it{Me aconsejó que se lo diese a ella.}

	\item El \textcolor{blue}{acto perlocutivo} es el que se realiza por haber dicho algo (\it{by saying something}) y se refiere a los actos producidos $\rightarrow$ logra efectos en el interlocutor. No siempre es fácilmente identificable.

	\it{Me convenció de que se lo diera a ella.}
\end{itemize}

\end{frame}

\begin{frame}{Consecuencias de la tricotomía locutivo/ilocutivo/perlocutivo}

	\begin{itemize}
		\item A partir de esta triconomía, es posible reexaminar el problema de las relaciones entre enunciados y acciones. 
		\item Reconocer que presencia simultánea de actos locutivo, ilocutivo y perlocutivo permite explicar que incluso los enunciados menos claramente realizativos tienen propiedades que los acercan a las acciones. 
		\item Todos los enunciados ---no solo los realizativos--- se prestan a infortunios.
		
		mala apelación: \it{Los hijos de Juan son rubios} (cuando sabemos que Juan no tiene hijos).

		mala ejecución: \it{En estos momentos de embargo, la emoción me jubila}.

		enunciado insincero: \it{Está lloviendo} (cuando sé a ciencia cierta que hace un sol espléndido).
	\end{itemize}

\end{frame}

\begin{frame}{Austin: Resumen}

	\begin{itemize}
		\item Las distinciones establecidas por Austin son fundamentales en todos los estudios posteriores sobre el significado y la Pragmática moderna.
		\item El lenguaje no es solo descriptivo. No todos los enunciados tienen que ser verdaderos o falsos.
		\item El estudio de los enunciados realizativos permite establecer un vínculo entre las palabras y las acciones. 
		\item La tricotomía locutivo/ilocutivo/perlocutivo.
	\end{itemize}

\end{frame}


\begin{frame}{}
\begin{center}
  \LARGE{Searle: La teoría de los actos de habla}
\end{center} 
\end{frame}

\begin{frame}{John Searle}

\begin{itemize}
	\item Los trabajos de John Searle continúan la línea de investigación iniciada por Austin.
	\item Searle es un filósofo del lenguaje estadounidense, profesor de la UC Berkeley.
	\item En algunos casos, sus ideas desarrollan los planteamientos de Austin, llevando hasta las últimas consecuencias las intuiciones de éste.
	\item Searle también está más cerca de la Filosofía del Lenguaje que de la Lingüística, pero sus teorías se popularizaron entre numerosos estudiosos del lenguaje.
\end{itemize}

\end{frame}

\begin{frame}{Puntos de partida de Searle}

\begin{itemize}
	\item El punto de partida de la teoría de Searle es el siguiente:
	
	\vspace{0.5cm}
	
	\it{Hablar una lengua es tomar parte en un forma de conducta (altamente compleja) gobernada por reglas. Aprender y dominar una lengua es haber aprendido y dominado tales reglas}.
	
	\vspace{0.5cm}

	\item En la práctica, esta hipótesis sugiere una extensión de las ideas de Austin.
	\item Sigue habiendo una clara distinción entre acción y lenguaje.
	\item En la comunicación, el uso del lenguaje se entiende como un tipo particular de acción.
	\item Toda actividad lingüística (y no solo cierto tipo de actos ritualizados, como sostenía Austin) es convencional y está controlada por reglas.
\end{itemize}

\end{frame}

\begin{frame}{La Teoría de los Actos de Habla}

\begin{itemize}
	\item El \textcolor{blue}{acto de habla} se define como la emisión de una oración hecha en las condiciones apropiadas.
	\item El acto de habla es la unidad mínima de la comunicación y se convierte en el centro de la teoría de Searle.
	\item Las oraciones (en cuanto unidades abstractas, teóricas, no realizadas) no pueden ser las unidades básicas de la comunicación humana porque carecen de la dimensión fundamental para ello: no han sido producidas. 
	
	\vspace{0.5cm}
	
	\it{Hablar una lengua consiste en realizar actos de habla, actos tales como hacer afirmaciones, dar órdenes, plantear preguntas, hacer promesas, etc; más abstractamente, actos tales como referir y predicar; [\ldots] estos actos son posibles gracias a, y se realizan de acuerdo con, ciertas reglas para el uso de los elementos lingüísticos}.
\end{itemize}

\end{frame}

\begin{frame}{La Teoría de los Actos de Habla}

	\begin{itemize}
		\item El lenguaje, o mejor dicho, el uso del lenguaje en la comunicación está sometido a una serie de reglas (no necesariamente conscientes) que gobiernan cualquier emisión lingüística.
		\item Los infortunios de los que hablaba Austin para los enunciados realizativos no serían más que un caso de diversos fallos  en la correcta aplicación de estas reglas. 
		\item Estos principios reguladores que utilizamos en el lenguaje no son diferentes, en esencia, a los que que seguimos en cualquier otro tipo de actividad humana. 
		\item Searle propone analizar las características formales de la oración emitida y las condiciones que deben darse para poder realizar con éxito un determinado tipo de acto.
	\end{itemize}

\end{frame}

\begin{frame}{Fuerza ilocutiva y forma lingüística}

\begin{itemize}
	\item Searle parte de la idea de acto ilocutivo de Austin: la fuerza, el poder de hacer el acto que se realiza al decir algo (\it{in saying something}).
	\begin{enumerate}
		\item Juan fuma habitualmente.
		\item ¿Juan fuma habitualmente?
		\item ¡Juan, fuma habitualmente!
		\item Ojalá Juan fumase habitualmente.
	\end{enumerate}
	\item Las cuatro frases anteriores contienen un mismo contenido proposicional `Juan fuma' aunque difieren de su \textcolor{blue}{fuerza ilocutiva}.
	\item Según Searle, la forma lingüística y la fuerza ilocutiva están íntimamente unidas por una relación regular y constante. 
	\item Esto implica que hay una relación sistemática entre la forma lingüística del imperativo y el acto de habla \it{mandato}, o entre la forma lingüística interrogativa y el acto de habla \it{pregunta}.
\end{itemize}
\end{frame}

\begin{frame}{Fuerza ilocutiva y forma lingüística}

\begin{itemize}
	\item La identificación extrema entre fuerza ilocutiva y forma lingüística tiene importantes consecuencias en la propia estructura de la teoría del lenguaje que hemos visto hasta ahora.
	\item Para Searle, cada uno de los tipos de acto de habla está convencionalmente asociado con una determinada estructura lingüística. 
	\item Una determinada estructura sintáctica se corresponderá siempre con un determinado tipo de acto de habla.
	
	\pause
	 
	\item Si la fuerza ilocutiva pasa a ser una parte constitutiva de la caracterización sintáctica de las estructuras oracionales, desaparece el división entre Semántica y Pragmática.
	\item De hecho, Searle aboga por eliminar la separación entre ambas disciplinas.
\end{itemize}

\end{frame}

\begin{frame}{Fuerza ilocutiva y forma lingüística}

	El significado de cualquier oración puede analizarse en dos partes:
	 
\begin{itemize}
	\item \textcolor{blue}{indicador proposicional}: contenido expresado por la proposición (en general, la unión entre una referencia y una predicación).
	\item \textcolor{blue}{indicador de fuerza ilocutiva}: muestra en qué sentido (con qué fuerza ilocutiva) debe interpretarse la proposición. 
	
	Entre estos indicadores de la fuerza ilocutiva podemos señalar la entonación, el énfais, el orden de las palabras y la presencia de ciertos verbos realizativos explícitos.  
\end{itemize}

\end{frame}

\begin{frame}{Fuerza ilocutiva y forma lingüística}

\begin{itemize}
	\item Formalmente, podemos representar esta bipartición como una función entre una fuerza ilocutiva (representada por $F$ y un contenido proposicional (representado por la variable $p$).
	
	$F(p)$
	
	\item La función ilocutiva puede tomar diferentes forma según sea el acto ilocutivo representado: 
		\begin{itemize}
			\item $\vdash$ aserción
			\item $Pr$ promesa
			\item $!$ petición
			\item $?$ pregunta		
		\end{itemize}
	\item De manera que $\vdash (p)$ representa la aserción de la proposición $p$, mientras que $Pr (p)$ representa una promesa de la proposición $p$. 
\end{itemize}

\end{frame}

\begin{frame}{Fuerza ilocutiva y forma lingüística}

	\begin{itemize}
		\item Como hemos visto antes, un mismo contenido proposicional puede utilizarse con distintas fuerzas ilocutivas.

		\vspace{0.2cm}
		
		\begin{tabular}{l c}
			Juan viene. &  $\vdash(VENIR, Juan)$ \\ 
			Prometo que Juan vendrá. & $Pr(VENIR, Juan)$ \\ 
			¡Que venga Juan! & $!(VENIR, Juan)$  \\
			¿Viene Juan? & $?(VENIR, Juan)$  \\
		\end{tabular}
		
		\item Esta notación permite diferenciar claramente entre la negación proposicional y la negación ilocutiva.

		\vspace{0.2cm}
		
		\begin{tabular}{l c}
			Prometo no venir. & $Pr(\neg VENIR, yo)$ \\ 
			No prometo venir. & $\neg Pr(VENIR, yo)$ \\ 
		\end{tabular}
	\end{itemize}

\end{frame}

\begin{frame}{Clasificación de actos ilocutivos}

	\begin{itemize}
		\item A pesar de la dificultad de enumerar los diferentes actos ilocutivos, Searle propone reducirlos a cinco tipos:
		\vspace{0.2cm}

		 \it{[\ldots] hay un número más o menos limitado de cosas que se hacen con el lenguaje: decimos a la gente cómo son las cosas (\textcolor{blue}{actos asertivos}); tratamos de conseguir que hagan cosas (\textcolor{blue}{actos directivos}); nos comprometemos a hacer cosas (\textcolor{blue}{actos compromisivos}); expresamos nuestros sentimientos y actitudes (\textcolor{blue}{actos expresivos}); y producimos cambios a través de nuestras emisiones (\textcolor{blue}{declaraciones}). A menudo, hacemos más de una de estas cosas a la vez.}
	\end{itemize}

\end{frame}

\begin{frame}{Condiciones de adecuación de los actos ilocutivos}

	Searle propone enunciar las condiciones que permiten la realización de un acto ilocutivo, p. ej., el de pedir:
	
	\begin{itemize}
		\item \textcolor{blue}{condiciones de tipo general}: las que hacen posible la comunicación. P. ej., hablar inteligiblemente.
		\item \textcolor{blue}{condiciones de contenido proposicional}: características significativas de la proposición empleada para llevar a cabo el acto de habla. P. ej., referirnos a un acto futuro del oyente.
		\item \textcolor{blue}{condiciones preparatorias}: condiciones que deben darse para que tenga sentido realizar el acto ilocutivo. P. ej., en la petición, que el oyente no parezca predispuesto a hacerlo espontáneamente. 
		\item \textcolor{blue}{condiciones de sinceridad}: condiciones centradas en el estado psicológico del hablante. P. ej., que desea sinceramente que su interlocutor haga lo que se le pide.
		\item \textcolor{blue}{condiciones esenciales}: caracterizan el acto, hacen que la petición sea una petición y no otro acto. P. ej. la emisión lingüística cuenta como un intento de que el oyente haga algo. 
	\end{itemize}

\end{frame}

\begin{frame}{Condiciones de adecuación de los actos ilocutivos}

	\begin{itemize}
		\item Con la enumeración de estas condiciones, Searle intenta generalizar el análisis de infortunios y fracasos de los enunciados realizativos de Austin. 
		\item Cuando las condiciones estipuladas no se cumplen en alguno de los aspectos, el resultado es también algún tipo de infortuno, cuya naturaleza y consecuencias varían dependiendo de la regla infringida.
		\item P. ej., una petición que no cumpla la condición de sinceridad provoca un infortunio diferente al que se produce al fallar las condiciones preparatorias.
		\item ¿Qué hace falta para encender una vela? \pause Que esté apagada.
	\end{itemize}

\end{frame}

\begin{frame}{Condiciones de adecuación de los actos ilocutivos}

	Pensemos en otro ejemplo de caracterización de un acto ilocutivo: la pregunta \it{¿Quién viene?}

	\begin{itemize}
		\item \textcolor{blue}{condiciones de tipo general}: hablar inteligiblemente.
		\item \textcolor{blue}{condiciones de contenido proposicional}: 'alguien viene', $viene(X)$.
		\item \textcolor{blue}{condiciones preparatorias}: el hablante desconoce la respuesta, y no es obvio que el oyente vaya a proporcionar la respuesta si no se le pregunta. 
		\item \textcolor{blue}{condiciones de sinceridad}: el hablante desea conocer esa información.
		\item \textcolor{blue}{condiciones esenciales}: la emisión del enunciado cuenta como un intento de obtener del oyente esa información. 
	\end{itemize}

\end{frame}

\begin{frame}{Actos de habla indirectos}

	\begin{itemize}
		\item Searle afirmaba que hay una correspondecia directa entre forma lingüística y el acto de habla: la forma lingüística del imperativo y el acto de habla \it{mandato}, o entre la forma lingüística interrogativa y el acto de habla \it{pregunta}.
		\item Un análisis como el anterior solo tiene sentido para las oraciones interrogativas usadas como peticiones de información, o en general, cuando los hablantes utilizan el lenguaje de manera literal.
		\item Cuando utilizamos el lenguaje de manera literal, la fuerza ilocutiva deriva directamente del conocimiento que tienen los oyentes de las reglas que gobiernan la emisión de las oraciones. 
		\item Sin embargo, en numerosas ocasiones, los hablantes quieren decir algo ligeramente distinto a lo que realmente expresan y usan el lenguaje de manera indirecta. Estos casos reciben el nombre de \textcolor{blue}{actos de habla indirectos}.
%		\item ¿Cómo explicamos estos casos?
	\end{itemize}

\end{frame}

\begin{frame}{Actos de habla indirectos}

		\it{¿Podrías pasarme la sal? Me gustaría que hicieras esto}.

	\vspace{0.3cm}
	
	\begin{itemize}
		\item En los ejemplos anteriores, tenemos una interrogativa y una asertiva-desiderativa y, a pesar de que reúnen las condiciones formales para ello, en la mayor parte de contextos su emisión no realizará los actos ilocutivos de pregunta o expresión de deseo.
		\item Estamos ante ejemplos de enunciados que realizan un acto ilocutivo diferente del que su forma lingüística haría prever.
		\item Searle explica los \textcolor{blue}{actos de habla indirectos} como la superposición de dos actos: uno literal y otro indirecto. 
		\item El oyente interpreta el verdadero acto de habla (el indirecto) gracias a su conocimiento del contexto y su capacidad para interpretar la intención del hablante al hacer la pregunta-petición.
	\end{itemize}

\end{frame}


\begin{frame}{Críticas al análisis de Searle}

	\begin{itemize}
		\item El análisis de los actos ilocutivos indirectos constituye el principal obstáculo con el que tropieza la teoría de los actos de habla. 
		\item Para que una oración tenga su sentido literal y realice el acto de habla esperado, debe emitirse en un contexto adecuado. 
		\item Si es así, es entonces el contexto de la emisión (y no su forma lingüística) lo que determina qué acto de habla realiza una determinada estructura oracional. 
		\item Resulta poco económico (desde el pdv teórico) afirmar que oración y acto de habla están indisolublemente unidos si luego hay que acabar concediendo que esta unión está siempre a merced del contexto.
		\item A pesar de estas y otras quejas, la importancia de la teoría de actos de habla de Searle está fuera de toda duda y ha enriquecido el estudio del uso efectivo del lenguaje. 
	\end{itemize}

\end{frame}

\begin{frame}{}
\begin{center}
  \LARGE{El modelo de Grice}
\end{center} 
\end{frame}

\begin{frame}{Puntos de partida}

\begin{itemize}
	\item Mientras que Austin situaba sus trabajos dentro de la Filosofía del Lenguaje y Searle abogaba por una eliminación de los límites entre el significado de una oración y su fuerza ilocutiva, Paul Grice sí es el primer autor en enmarcar sus contribuciones dentro del campo de la Pragmática.
	\item Grice es un filósofo británico del siglo XX, falleció en 1988 cuando todavía era profesor en UC Berkeley. 
	\item Su trabajo, recopilado en la obra \it{Studies in the Way of Words}, ha tenido una gran importancia en filosofía y lingüística, con implicaciones también en el ámbito de la ciencia cognitiva en general.
\end{itemize}

\end{frame}

\begin{frame}{El Principio de Cooperación}

\begin{itemize}
	\item Grice propone un análisis del tipo particular de lógica que actúa y rige en la conversación.
	\item Como hemos visto anteriormente, Searle había propuesto determinadas condiciones que permitían la realización de un acto ilocutivo. 
	\item Grice, por el contrario, propone una serie de principios, no prescriptivos sino descriptivos, que se suponen aceptados tácitamente por todos los participantes en una conversación: el \textcolor{blue}{principio de cooperación}.
	\item Cuando los participantes no se ajustan a él, la conversación es inconexa y absurda y pueden acarrear algún tipo de sanción social.
	\item Si uno de los interlocutores boicotea la conversación deliberadamente se expone a ser excluído del diálogo.
\end{itemize}

\end{frame}

\begin{frame}{El Principio de Cooperación y las máximas}

	Grice organiza su principio de cooperación en cuatro categorías, cada una de ellas dividida a su vez en distintas máximas.
	
\begin{itemize}
	\item \textcolor{blue}{Cantidad}: se relaciona con la cantidad de información que debe proporcionarse.
	\begin{itemize}
		\item a) que su contribución sea todo lo informativa que requiera el propósito del diálogo.
		\item b) que su contribución no sea más informativa de lo necesario.
	\end{itemize}

	\vspace{0.3cm}
	
	\item \textcolor{blue}{Cualidad}: se relaciona con el grado de verdad de la información. «Intente que su contribución sea verdadera».
	\begin{itemize}
		\item a) no diga algo que crea falso.
		\item b) no diga algo de lo que no tenga pruebas suficientes.
	\end{itemize}

\end{itemize}

\end{frame}

\begin{frame}{El Principio de Cooperación y las máximas}

\begin{itemize}
	\item \textcolor{blue}{Relación}: se espera de los participantes en la conversación que sus intervenciones se relacionen con aquello de los que se está hablando. «Diga cosas relevantes».

	\vspace{0.3cm}
	
	\item \textcolor{blue}{Modalidad}: se relaciona con el modo de decir las cosas. «Sea claro».
	\begin{itemize}
		\item a) evite la oscuridad en la expresión.
		\item b) evite la ambigüedad.
		\item c) sea breve (no sea innecesariamente prolijo)
		\item d) sea ordenado.
	\end{itemize}
\end{itemize}

\end{frame}

\begin{frame}{El Principio de Cooperación y las máximas}

\begin{itemize}
	\item Nuestros intercambios comunicativos corrientes no consisten en una mera sucesión de observaciones desconectadas.
	\item La conversación comporta, normalmente, un esfuerzo por colaborar con nuestro interlocutor. 
	\item Debemos comportarnos así porque es lo que los demás esperan de nosotros, y nosotros de los demás.
	\item Tan fuerte es esa expectativa que, si el hablante parece no cumplir con el principio de cooperación, el oyente, en lugar de pensar que efectivamente el hablante no cumple, pensará que en realidad quiere decir otra cosa.
	\item Esa otra cosa (la \textcolor{blue}{implicatura}) será un significado adicional comunicado por el hablante e inferido por el oyente.
	\item Esta pirueta de la comunicación (comunicar sin decir, contando con que el oyente va a inferir lo implicado) es posible siempre y cuando los interlocutores descuenten el cumplimiento del principio de cooperación.
\end{itemize}

\end{frame}


\begin{frame}{Tipos de contenidos implícitos}

	\begin{itemize}
		\item Grice establece una diferencia neta entre: 
		\begin{itemize}
			\item \textcolor{blue}{Lo que se dice} se corresponde básicamente con el contenido proposicional del enunciado, tal y como se entiende desde el pdv lógico, y se puede evaluar dentro de la lógica veritativa.
			\item \textcolor{blue}{Lo que se comunica} es toda la información que se transmite con el enunciado pero que es diferente de su contenido proposicional. 
		\end{itemize}
		\item Este contenido implícito recibe el nombre de \textcolor{blue}{implicatura}.
Las implicaturas pueden surgir para tender el puente necesario entre la aparente violación de una máxima y la presunción de que el principio de cooperación sigue vigente.
		\item Las implicaturas se producen en los siguientes casos: a) cuando el hablante obedece a las máximas; b) cuando parece violarlas pero no las viola; c) cuando viola una máxima para no violar otra de mayor importancia; y d) cuando viola una máxima abierta y deliberadamente.
	\end{itemize}

\end{frame}

\begin{frame}{Detección de implicaturas}

	\begin{itemize}
		\item a) \textcolor{blue}{obediencia a las máximas}: El oyente saca la implicatura correspondiente asumiento que su interlocutor respeta las máximas y el principio de cooperación.
		
		P. ej. Está lloviendo a la salida del trabajo y le digo a un compañero: \it{Yo tengo coche}.
		
		\item b) \textcolor{blue}{violación aparente}: A veces el hablante parecer violar las máximas, lo que genera implicaturas cuando el oyente asume que su interlocutor las respeta. Hay implicatura, pero no hay violación aparente.

		P. ej. al solicitar una carta de recomendación para entrar en el equipo de baloncesto de la universidad, un entrenador escribe:
		
		\it{El Sr. X asiste siempre a clase, hace puntualmente todos los trabajos que se le asignan y se expresa con propiedad.}

	\end{itemize}

\end{frame}

\begin{frame}{Detección de implicaturas}

	\begin{itemize}
		\item c) \textcolor{blue}{choques entre máximas}: Obliga a elegir entre una máxima en detrimento de otras. Se viola una máxima para evitar un conflicto con otra máxima. Típica de aquellas situaciones en las que no podemos proporcionar toda la información sin mentir.
		
		P. ej. al preguntar por la biblioteca, respondo \it{Quedá más al sur, pasando el parque}.
		
		\item d) \textcolor{blue}{violación ostentosa}: Calculamos implicaturas cuando advertimos que nuestro interlocutor está violando las máximas con deliberación. Hay implicatura precisamente porque se viola abiertamente una máxima. 

		P. ej. \it{-- ¿Cuándo comemos, mamá?}
		
		\it{-- Cuando esta señora que ahora está tratando de escribir un email al oportuno de su jefe se levante de la silla, vaya a la cocina y caliente la cena que lleva hecha desde las seis\ldots}

	\end{itemize}

\end{frame}

\begin{frame}{Las implicaturas}

	Existen dos grandes tipos de implicaturas. Por un lado:
	
	\begin{itemize}
		\item \textcolor{blue}{implicaturas convencionales}: son aquellas que derivan directamente de los significados de las palabras, y no de factores contextuales o situacionales.

		\vspace{0.5cm}
				
		En el ejemplo \it{Era pobre pero honrado} se genera una implicatura convencional ligada al significado de \it{pero}. El segundo adjetivo se presenta como un contraste con respecto al primero y como algo poco habitual o inesperado.
		
	\end{itemize}

\end{frame}

\begin{frame}{Las implicaturas}

	Por otro lado:
	
	\begin{itemize}
		\item \textcolor{blue}{implicaturas no convencionales}: son aquellas que se generan por la intervención interpuesta de otros principios, y distinguimos:
		
		\begin{itemize}
			\item \textcolor{blue}{conversacionales}: cuando invocamos los principios que regulan la conversación, es decir, el principio de cooperación y las máximas que lo desarrollan.
			\begin{itemize}
				\item \textcolor{blue}{particularizadas}: las que dependen decisivamente del contexto de emisión.
				\item \textcolor{blue}{generalizadas}: las que no dependen directamente del contexto. 
			\end{itemize}
			\item \textcolor{blue}{no conversacionales}: cuando los principios en juego son de tipo ético, social o moral.
		\end{itemize}
		
	\end{itemize}

\end{frame}

\begin{frame}{Las implicaturas}

	El significado se divide entre:
	
	\begin{itemize}
		\item Lo dicho.
		\item Las \textcolor{blue}{implicaturas} (lo implícito, lo comunicado).
		\begin{itemize}
			\item \textcolor{blue}{convencionales}
			\item no convencionales
			\begin{itemize}
				\item \textcolor{blue}{conversacionales}: \textcolor{blue}{particularizadas} / \textcolor{blue}{generalizadas}
				\item no conversacionales
			\end{itemize}
		\end{itemize}
	\end{itemize}

\end{frame}

\begin{frame}{Tipos de implicaturas}

	\begin{itemize}
		\item Grice distingue, principalmente, tres tipos de implicaturas: convencionales, conversacionales particularizadas y conversacionales generalizadas. 
		\item Para distinguirlas, Grice tiene en cuenta en qué medida cumplen cuatro propiedades:
		\item 1) \textcolor{blue}{convencionalidad}: frente a conversacionalidad.
		\item 2) \textcolor{blue}{calculabilidad}: se aplica a implicaturas que resultan de un proceso de inferencia por parte del oyente.
		\item 3) \textcolor{blue}{cancelabilidad}: se aplica a las implicaturas que pueden eliminarse al cambiar de contexto.
		\item 4) \textcolor{blue}{separabilidad}: se aplica a las implicaturas que desaparecen al enunciar la expresión de otra manera.
	\end{itemize}

\end{frame}

\begin{frame}{Implicaturas convencionales y presuposiciones}

	\begin{itemize}
		\item Las \textcolor{blue}{implicaturas convencionales} forman parte del contenido de ciertas expresiones lingüísticas. Son significados adicionales que vienen incorporados en el contenido de las expresiones.
		\item Son no calculables, no cancelables y separables.
		
		\item El enunciado \it{María logró terminar la tesis} contiene la proposición `María terminó la tesis' y, además, añade que le costó algún esfuerzo terminarla, por la razón que fuera.
		
		\item La idea de esfuerzo o dificultad está implícita en la construcción \it{lograr + infinitivo}. No requiere ningún contexto específico, está adherida a la constucción lingüística.
		
		\item El enunciado \it{María terminó la tesis} significa estrictamente lo mismo. La proposición básica es la misma, pero en este caso no hay significado extra acerca del esfuerzo que le costó a María.
	\end{itemize}

\end{frame}

\begin{frame}{Implicaturas convencionales y presuposiciones}

	\begin{itemize}
		\item También son convencionales un tipo especial de implicaciones estudiadas desde hace años en Semántica: las presuposiciones.
		\item Las \textcolor{blue}{presuposiciones} son significados adicionales que están implícitos en ciertas expresiones y que cuentan para evaluar la verdad de la oración.
		\item P. ej., en el enunciado \it{Gutiérrez dejó de llamarme} hay una presuposición: `Gutiérrez me llamaba', que debe ser cierta para que el enunciado lo sea.
		\item La presuposición subsiste incluso si se niega la oración: \it{Gutiérrez no dejó de llamarme}.
		\item Las presuposiciones se distinguen de las implicaturas convencionales en que las primeras no son separables. 
		
		\it{Luis dejó de fumar} / \it{Luis ya no fuma} / \it{Luis no fuma más} presuponen que `antes fumaba'.
		
		\it{Son pobres pero honrados} $\iff$ \it{Son pobres y honrados} 		
	\end{itemize}

\end{frame}

\begin{frame}{Implicaturas conversacionales particularizadas}

	\begin{itemize}
		\item Las \textcolor{blue}{implicaturas conversacionales particularizadas} son calculables, cancelables y no separables.
		\item Precisamente porque dependen de un contexto específico, estas implicaturas pueden cancelarse. En el ejemplo de la carta de recomendación, bastaría que la recomendación fuese para ingresar en algún puesto que requiriera las cualidades descritas en la carta.
		\item Estas implicaturas son no separables porque no están adheridas a ninguna expresión lingüística. Si la carta dijera lo mismo con otras palabras: \it{El Sr. X es aplicado, puntual y escribe con gran corrección} la implicatura `no sirve para el equipo' seguiría vigente.
	\end{itemize}

\end{frame}

\begin{frame}{Implicaturas conversacionales generalizadas}

	\begin{itemize}
		\item Las \textcolor{blue}{implicaturas conversacionales generalizadas} también son calculables, cancelables y no separables, pero no dependen de un contexto específico.
		\item En \it{Marta va a encontrarse con un hombre esta noche}, \it{un} implica que el hombre no es su marido, ni un miembro de su familia. 
		\item En \it{Entré en una casa}, \it{una} indica que no entré en mi casa.
		\item El artículo indeterminado implica que no hay relación cercana entre la entidad a la que se refiere el artículo y el individuo pertinente del contexto.
	\end{itemize}

\end{frame}

\begin{frame}{Implicaturas conversacionales generalizadas}

	\begin{itemize}
		\item Otro ejemplo que provoca implicaturas conversacionales generalizadas es la observación de la máxima de cantidad.
		\item \it{Julia escribió tres libros} implica que `escribió tres libros', no dos, ni cinco.
		\item Pero ese significado de `exactamente tres' es una implicatura, ya que la expresión significa estrictamente `por lo menos tres'. Si Julia escribió siete libros, sigue siendo verdad que escribió tres.
		\item En la conversación, sería muy poco cooperativo decir que escribió tres si sabemos que escribió más de tres. La lógica propia de la conversación permite desambiguar las expresiones lingüísticas.
		\item La diferencia entre implicaturas conversacionales particularizadas y generalizadas hay, solamente, una diferencia de grado en cuanto a su dependencia del contexto. 
	\end{itemize}

\end{frame}

\begin{frame}{Implicaciones pragmáticas: resumen}

	\begin{itemize}
		\item \textcolor{blue}{presuposición}: no calculable, no cancelable, no separable: \it{Juan dejó de fumar}.
		\item \textcolor{blue}{implicatura convencional}: no calculable, no cancelable, separable: \it{Es pobre pero honrado}.
		\item \textcolor{blue}{implicatura conversacional particularizada}: calculable, cancelable, no separable: \it{X es aplicado y puntual}.
		\item \textcolor{blue}{implicatura conversacional generalizada}: calculable, cancelable, no separable: \it{Entré en una casa}.
		
	\end{itemize}
\end{frame}

\begin{frame}{}
\begin{center}
  \LARGE{La Teoría de la Relevancia: Sperber y Wilson}
\end{center} 
\end{frame}

\begin{frame}{El concepto de Relevancia}

	\begin{itemize}
		\item La Teoría de la Relevancia, formulada por Dan Sperber y Deirdre Wilson en \it{Relevance. Communication \& Cognition} y en otras obras conjuntas, es uno de los modelos más influyentes y atractivos del panorama general de la Pragmática.
		\item Sperber es un lingüista y científico cognitivo francés. Wilson es una psicóloga del lenguaje británica. 
		\item La Teoría de la Relevancia se alinea con otras teorías que ponen el énfasis en la idea de que no hay una correspondencia biunívoca constante entre las representaciones semánticas abstractas de las oraciones y las interpretaciones concretas de los enunciados: lo que decimos y lo que queremos decir no siempre coinciden. 
		\item La Teoría de la Relevancia está basada en las ideas de Grice aunque propone una manera diferente de explicar el proceso de la comunicación lingüística.
	\end{itemize}

\end{frame}

\begin{frame}{La Relevancia}

	\begin{itemize}
		\item La \textcolor{blue}{relevancia} es el principio que explica todos los actos comunicativos lingüísticos. 
		\item Nuestro interlocutor es relevante (pertinente) y lo que comunica también lo es, por eso le prestamos atención.
		\item ¿Por qué somos cooperativos en los actos comunicativos? Principalmente porque tenemos algo que ganar: conocimiento del mundo.
		\item Cada enunciado lingüístico intencional viene con una garantía de relevancia. 
		\item Todas nuestras actividades informativas se orientan hacia la meta (general y abstracta) de mejorar nuestro conocimiento del mundo.
		\item Tenemos la expectativa de que nuestro interlocutor trate de ser relevante y que contribuya a enriquecer nuestro conocimiento del mundo, sin exigirmos un esfuerzo desmedido de interpretación.
	\end{itemize}

\end{frame}


\begin{frame}{La Relevancia}

	\begin{itemize}
		\item Si una persona produce un estímuno verbal deliberado, este estímulo merece nuestra atención y el esfuerzo de interpretarlo, ya que producirá efectos cognoscitivos que nos interesan, a corto o a largo plazo.
		\item Equilibramos ganancia (adquisición de información, de nuevo conocimiento del mundo) y esfuerzo (de procesamiento para interpretar los enunciados que recibimos).

		\vspace{0.3cm}
		$Relevancia = \frac{Efectos \ cognoscitivos}{Esfuerzo \ de \ procesamiento}$ 
		\vspace{0.3cm}
		\item La relevancia no es un concepto absoluto: hay grados.
		\item Cuanto más efectos cognoscitivos produzca un enunciado y menos esfuerzo de interpretación exija, más relevante será.
	\end{itemize}

\end{frame}


\begin{frame}{La Relevancia}

	En nuestro entorno cognoscitivo, podemos encontrar tres tipos diferentes de información:
	
	\begin{enumerate}
		\item información inmediatamente accesible que no necesita ser procesada.
		\item información desconectada que requiere mucho esfuerzo de procesamiento.
		\item información nueva pero conectada con lo que ya tenemos: la propia conexión genera más información nueva 	que no se habría podido inferir si no fuera por la conexión.
	\end{enumerate}
	
	\begin{itemize}
		\item Este tercer tipo de información es la más relevante, pues produce un efecto de multiplicación con menos coste de procesamiento. 
	\end{itemize}

\end{frame}


\begin{frame}{El Principio de Relevancia}

	\begin{itemize}
		\item El Principio de Relevancia es cognoscitivo y universal.
		\item Las máximas de Grice resultan superfluas y asumimos que la conducta lingüística está fundada sobre la relevancia.
		\item La relevancia basta para explicar la comunicación lingüística: se aplica a todos los actos comunicativos intencionales, sin excepción.
		\item El Principio de Relevancia no se puede transgredir ni violar.
		\item Por supuesto, un hablante puede fracasar en su intento comunicativo y no ser relevante. Pero es suficiente con que transmita, a través de la producción de un enunciado, la presunción de que es relevante.
	\end{itemize}

\end{frame}

\begin{frame}{La Pragmática para Sperber y Wilson}

	\begin{itemize}
		\item Para Sperber y Wilson, el objeto de la Pragmática es la teoría de interpretación de los enunciados.
		\item Otorgan un papel fundamental a la \textcolor{blue}{inferencia}.
		\item Entender un enunciado consiste en dos pasos.
	\end{itemize}
	\begin{enumerate}
		\item descodificar signos lingüísticos
		\item superar el escalón entre lo dicho y lo implicado a través de inferencias.
	\end{enumerate}

\end{frame}

\begin{frame}{La inferencia}

	\begin{itemize}
		\item La \textcolor{blue}{inferencia} es un proceso de razonamiento deductivo.
		\item A partir de determinadas premisas, se llega a una conclusión que se deriva lógicamente.
		
		\vspace{0.3cm}
		A: -- ¿Vas a comprar el diccionario?
		
		B: -- Gasté todo el dinero.
		\vspace{0.3cm}
		
		\item A construye un contexto con conocimientos y creencias. A sabe que B quiere ser relevante, y la única interpretación posible es que B no puede comprar el diccionario.
		\item A y B comparten una versión parecida del contexto denominado \textcolor{blue}{conocimiento mutuo}: lo que cada interlocutor sabe y sabe que el otro sabe.
	\end{itemize}

\end{frame}

\begin{frame}{Las explicaturas}

	\begin{itemize}
		\item Grice distinguió dos niveles de significado:
	\end{itemize}
	
	\begin{enumerate}
		\item lo dicho: aquella parte de la comunicación que se puede evaluar según el criterio de verdad.
		\item lo comunicado (las implicaturas): la información implícita que el destinatario extrae del enunciado.
	\end{enumerate}
	
	\begin{itemize}
		\item Hay elementos de lo dicho que no están determinados por el contenido semántico de la oración y solo adquieren significado en el preciso instante en que se emiten en una situación concreta de habla.
	\end{itemize}

\end{frame}

\begin{frame}{Las explicaturas}

	\begin{itemize}
		\item En la Teoría de la Relevancia, se distinguen tres niveles de significado:
	\end{itemize}
	
	\begin{enumerate}
		\item el significado convencional.
		\item lo dicho: las explicaturas.
		\item lo comunicado: las implicaturas.
	\end{enumerate}
	
	\begin{itemize}
		\item Grice en centraba en el paso del nivel 2 al nivel 3: la extracción de significados implícitos. 
		\item El nivel 3 resulta del proceso de descodificación e inferencia, incluyendo la inferencia de las implicaturas conversacionales. 
		\item La Teoría de la Relevancia se centra en cómo llegamos a interpretar el nivel 2, proponiendo que el paso de un nivel a otro se cumple mediante un proceso inferencial similar al de Grice. 
	\end{itemize}

\end{frame}


\begin{frame}{Las explicaturas}

	\begin{itemize}
		\item Lo dicho es la parte del significado que no se manifiesta explícitamente sino que es el resultado de procesos de desambiguación, asignación de referencias y enriquecimiento a través de inferencias. 
		\item Así, podemos definir la \textcolor{blue}{explicatura} como la proposición completa que expresa el hablante y que resulta de tomar el significado convencional y sumar todo el proceso de enriquecimiento del significado: 
		\begin{itemize}
			\item significado convencional
			\item asignación de referencias
			\item desambiguación
			\item enriquecimiento de expresiones
		\end{itemize}
		\item Solo una vez que el el destinatario tiene una proposición completa puede extraer las implicaturas que correspondan, si es que existen.
		\item Este análisis amplía el campo de la Pragmática proporcionando una explicación de cómo entendemos el significado explícito.
	\end{itemize}

\end{frame}



\begin{frame}{Referencias}

\begin{itemize}
	\item Austin, J. L. \it{Cómo hacer cosas con las palabras}.
	\begin{footnotesize}\texttt{http://books.google.es/books?id=RiGNvAjLTRsC}\end{footnotesize}
	\item Escandell, M. V. \it{Introducción a la Pragmática}. Ariel Lingüística. 2006.
	\begin{footnotesize}\texttt{http://books.google.es/books?id=p33Kdy\_\_MnYC}\end{footnotesize}
	\item Moreno-Torres, I. \it{Lingüística para Logopedas}. Ediciones Aljibe. 2004.
	\item Reyes, G. \it{El abecé de la Pragmática}. Arco Libros. 1996.
	\begin{footnotesize}\texttt{http://books.google.es/books?id=ddPlAAAAMAAJ}\end{footnotesize}
	\item Reyes, G., Baena, E., Urios, E. \it{Ejercicios de Pragmática (II)}. Arco Libros. 1996.
	\item Yule, G. \it{El lenguaje}. Cambridge University Press. 1998.
	\begin{footnotesize}\texttt{http://books.google.es/books?id=eDlr9ALBmnUC}\end{footnotesize}
\end{itemize}

\end{frame}

\end{document}

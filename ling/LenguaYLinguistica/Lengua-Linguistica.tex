\documentclass[handout]{beamer}
%\documentclass{beamer}
\usetheme{default}
\usepackage[spanish]{babel}
\usepackage[utf8]{inputenc}

\usepackage{pgfpages}
\pgfpagesuselayout{2 on 1}[a4paper,border shrink=5mm]


% items enclosed in square brackets are optional; explanation below
\title[Comunicación]{Lengua y Lingüística}
%\subtitle[Signo]{El Signo}
\author[V. Peinado]{Víctor Peinado}
\institute[UCM]{
  \texttt{v.peinado@filol.ucm.es}\\[1ex]
  
  Grado de Logopedia, Universidad Complutense de Madrid\\[1ex]
}

\date[Noviembre 2011]{17 de octubre de 2011}

\begin{document}

%--- the titlepage frame -------------------------%
\begin{frame}[plain]
  \titlepage
\end{frame}

%--- the presentation begin here -----------------%

\begin{frame}{La lingüística: objeto de estudio}

\begin{itemize}
	\item La Lingüística es la ciencia que se encarga del estudio de las lenguas.
	\item Es difícil establecer el número de lenguas que existen en el mundo ---ya veremos por qué---, pero se suele hablar de entre 3.000 y 7.000.
	\item Los estudios lingüísticos se suelen dividir en subdisciplinas según qué lenguas o qué partes de la lengua se consideran:
	\begin{itemize}
		\item centrados en una lengua o conjunto de lenguas: estudios hispánicos, estudios germánicos, estudios indoeuropeos, filología románica, filología eslava\ldots
		\item centrados en una subdisciplina o en las conexciones entre lingüística y otras disciplinas: fonética y fonología, morfología, lexicología y lexicografía, sintaxis, semántica, pragmática, dialectología, sociolingüística, neurolingüística, lingüística computacional\ldots
	\end{itemize}
\end{itemize}

\end{frame}

\begin{frame}{Subdisciplinas lingüísticas}

\begin{itemize}
	\item la \textcolor{blue}{fonética} y \textcolor{blue}{fonología} se ocupan de los sonidos.
	\item lo que tradicionalmente se conoce como \textcolor{blue}{gramática}, engloba:
	\begin{itemize}
		\item la \textcolor{blue}{morfología} se ocupa de los componentes y la forma de las palabras.
		\item la \textcolor{blue}{sintaxis} se ocupa de la estructura de la oración.
	\end{itemize}
	\item la \textcolor{blue}{lexicología} y la \textcolor{blue}{lexicografía} se ocupan del vocabulario de una lengua y de la elaboración de diccionarios.
	\item la \textcolor{blue}{semántica} se ocupa del significado de las palabras, los sintagmas y las oraciones.
	\item la \textcolor{blue}{pragmática} se ocupa de los significados del lenguaje en uso.

\end{itemize}

\end{frame}

\begin{frame}{Métodos de investigación en Lingüística}

\begin{itemize}
	\item Las lenguas están vivas: cambian, evolucionan a lo largo del tiempo y, en ocasiones, mueren y desaparecen.
	\item Teniendo en cuenta esto, podemos establecer dos tipos de estudios lingüísticos:
	\begin{itemize}
		\item \textcolor{blue}{sincrónico} o \textcolor{blue}{descriptivo}: describe la lengua en un momento concreto de su evolución, ya sea el estado presente o un momento concreto del pasado.
		\item \textcolor{blue}{diacrónico} o \textcolor{blue}{histórico}: describe la evolución de la lengua a lo largo del tiempo.
	\end{itemize}
	\item Las lenguas se pueden agrupar en familias lingüísticas y esto permite dos tipos de estudios:
	\begin{itemize}
		\item \textcolor{blue}{comparativo}: estudia una familia de lenguas atendiendo a su parentesco.
		\item \textcolor{blue}{tipológico}: estudia un conjunto de lenguas sin atender a su parentesco, sino a sus características estructurales.
	\end{itemize}
	
	
\end{itemize}


\end{frame}

\begin{frame}{Lingüística General}

\begin{itemize}
	\item Resume y generaliza conocimientos obtenidos del estudio de lenguas particulares.
	\item Trata de formular leyes que rigen una lengua o su evolución.
	\item Busca fenómenos comunes a todas las lenguas (los llamados \textcolor{blue}{universales lingüísticos}).
\end{itemize}
\end{frame}

\begin{frame}{Universales lingüísticos}
A pesar de la gran cantidad y variedad de lenguas que existen en el mundo hay dos factores que
llaman poderosamente la atención:

\begin{enumerate}
	\item cualquier lengua puede traducirse a otra y viceversa.
	\item cualquier recién nacido, independientemente de su origen y adscripción étnica, aprende la lengua de su comunidad.
\end{enumerate}

Estos dos argumentos permiten hipotizar la existencia de algunos elementos y rasgos comunes, un conjunto de propiedades compartidas por las diferentes lenguas del mundo que marcan los límites de nuestra facultad comunicativa: los \textcolor{blue}{universales lingüísticos}.

\end{frame}

\begin{frame}{Tipos de universales lingüísticos}

\begin{enumerate}
	\item los universales de tipo general, que se corresponden con las características del lenguaje propuestas por Hockett (semanticidad, arbitrariedad, desplazamiento, dualidad, productividad, disimulación y reflexividad).
	\item los universales de tipo estructural que afectan a diferentes niveles de estudio lingüístico:
\end{enumerate}

\begin{itemize}
	\item Nivel fonético-fonológico: 
	\begin{itemize}
		\item En todas las lenguas el sonido es articulado (la emisión del sonido pasa por articulaciones precisas y la intervención de órganos del aparato fonatorio). 
		\item Todas las lenguas tienen sonidos vocálicos y consonánticos.
	\end{itemize}
\end{itemize}

\end{frame}

\begin{frame}{Tipos de universales lingüísticos}

\begin{itemize}
	\item Nivel morfo-sintáctico: 
	\begin{itemize}
		\item Todas las lenguas tienen pronombres personales y deícticos.
		\item Todas las lenguas diferencian entre nombres y verbos.
		\item Todas las lenguas presentan construcciones basadas en núcleo + complemento.
		\item Todas las lenguas tienen oraciones aseverativas e interrogativas, construcciones
afirmativas y negativas. 
		\item Todas las lenguas tienen marcas temporales.
	\end{itemize} 
		  
	\item Nivel semántico: 
	\begin{itemize}
		\item Todas las lenguas tienen nombres comunes y nombres propios, palabras con
referentes concretos y abstractos. 
 		\item Todas las lenguas tienen palabras polisémicas, sinónimos, términos de parentesco, términos de color, etc.
	\end{itemize}
\end{itemize}

\end{frame}

\begin{frame}{Relaciones de la lingüística con otras ciencias}

\begin{itemize}
	\item la lingüística forma parte de lo que se conoce como ciencias sociales. 
	\item psicología, sociología, antropología, historia y filosofía. 
	\item existe una relación estrecha con los estudios de literatura $\rightarrow$ filología
	\item Pero también existe una relación estrecha con la física (acústica)
	\item y, recientemente, relación con ciencias exactas como las matemáticas
	\item la biología y la medicina
	\item y también con la informática y la teoría de la comunicación\ldots
\end{itemize}

\end{frame}




\begin{frame}{Lenguaje, lengua, habla}

\begin{itemize}
	\item Hasta ahora hemos hablado principalmente de \textcolor{blue}{lenguaje} y lo podemos definir como la capacidad humana de comunicarse y hacerse entender con ayuda de un conjunto de señales.
	
	\item A partir de aquí, distintos lingüistas han establecido una clara distinción entre la vertiente social y el carácter personal del lenguaje.
	
	P. ej., Ferdinand de Saussure distingue:
	\begin{itemize}
		\item lengua (\textit{langue}) es un sistema de signos y reglas compartido por una comunidad de hablantes.
		\item habla (\textit{parole}) es el acto de comunicación concreto formado a partir del conocimiento de la lengua.
	\end{itemize}
	
	\item Otros autores hablan incluso de \textcolor{blue}{idiolecto} para referirse a la lengua del individuo.
\end{itemize}


\end{frame}

\begin{frame}{Lengua}

\begin{itemize}
	\item La función fundamental de la lengua es comunicar (función comunicativa).
	\item La lengua es un fenómeno social/nacional.
	\item La lengua es un instrumento del pensamiento: permite generalizar y transmitir ideas.
	\item La lengua es un almacén de las experiencias de una comunidad, de transmisión de tradiciones y creación de cultura: es un medio de progreso de la humanidad.
	\item La lengua se puede utilizar para ejercer influencia sobre la gente.
\end{itemize}

\end{frame}

\begin{frame}{Referencias}

\begin{itemize}
   \item Bernárdez, E. \textit{¿Qué son las lenguas?} Alianza Ensayo. 2004.
   \item Tusón Valls, J. \textit{Introducción al lenguaje}. UOC. 2003.
   \item Yule, G. \textit{El lenguaje}. Ediciones AKAL. 2007. 
\end{itemize}

\end{frame}
\end{document}

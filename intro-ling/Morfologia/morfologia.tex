\documentclass{beamer}
%\documentclass[handout]{beamer}
\usetheme{default}
\usepackage[spanish]{babel}
\usepackage[utf8]{inputenc}
\usepackage{qtree}
\usepackage{amsmath}
\usepackage{tipa}
%\usepackage{multirow}

%\usepackage{pgfpages}
%\pgfpagesuselayout{2 on 1}[a4paper,border shrink=5mm]


% items enclosed in square brackets are optional; explanation below
\title[]{Morfología}
%\subtitle[Signo]{El Signo}
\author[V. Peinado]{Víctor Peinado}
\institute[UCM]{
  \texttt{v.peinado@filol.ucm.es}\\[1ex]
  
  Grado de Logopedia, Universidad Complutense de Madrid\\[1ex]
}

\date[Diciembre 2010]{20 de diciembre de 2010 - 17 de enero de 2011}

\begin{document}

\let\ipa\textipa
\let\it\textit


%--- the titlepage frame -------------------------%
\begin{frame}[plain]
  \titlepage
\end{frame}

%--- the presentation begin here -----------------%

\begin{frame}[Parte 1]

\begin{center}
  \LARGE{Las palabras: qué son y cómo se forman}
\end{center} 
\end{frame}


\begin{frame}{Etimología}

\begin{itemize}
    \item El estudio del origen y la historia de las palabras se conoce con el nombre de \textcolor{blue}{etimología}.
    \item Cuando se examina la etimología de las palabras, se advierte rápidamente que es posible incorporar nuevas palabras a una determinada lengua a través de diversas formas. 
    \item La incorporación de nuevas palabras se realiza lentamente a lo largo el tiempo. Todas las incorporaciones suelen considerarse durante un timpo barbarismos.
    \item No debemos criticar estos procesos como si se tratase de una degradación de la lengua, sino como un signo de vitalidad de las lenguas. Cualquier idioma se va remodelando para ajustarse a las necesidades de sus hablantes.
\end{itemize}

\end{frame}


\begin{frame}{Acuñación}

\begin{itemize}
    \item La invención de nuevas palabras se conoce con el nombre de \textcolor{blue}{acuñación}.
    \item Es uno de los procesos menos frecuentes para formar nuevas palabras.
    \item Suele tratarse de nombres de marcas comerciales creados a propósito para un determinado producto que, con el tiempo, terminan convirtiéndose en términos de uso general.
    
    Ejemplos: \it{aspirina, kleenex, nailon, rímel}
    
    \item Las palabras de nueva creación que derivan de nombres propios se denominan \textcolor{blue}{epónimos}.

    Ejemplos: \it{hoover, sandwich, jeans, farenheit, voltio, vatio}
\end{itemize}

\end{frame}

\begin{frame}{Préstamo}

\begin{itemize}
    \item Una de las formas más comunes de introducir un nuevas palabras en un idioma es tomarlas prestadas de otras lenguas.
    
    \item germánico y alemán: \it{sable, blanco, espuela, guante, guerra, cobalto, cuarzo, obús, zepelín}, etc.

    \item árabe: \it{alcohol, alcalde, almohada, guitarra, zanahoria}, etc.
    
    \item francés: \it{cruasán, chalet, amateur, chófer, jamón}, etc.
    
    \item italiano: \it{piano, batuta, soneto, novela, violín}, etc.

    \item americanismos: \it{chocolate, tiburón, jaguar, hamaca, canoa}, etc.

    \item inglés: \it{bistec, fútbol, gol, chutar, aparcar, hobby}, etc.

    \item turco: \it{yogurt}
    
    \item bantú: \it{cebra}
    
    \item checo: \it{robot}
\end{itemize}

\end{frame}

\begin{frame}{Préstamo}

\begin{itemize}
    \item De manera similar, encontramos ejemplos de otras lenguas han incorporado préstamos provenientes del castellano:
    
    Ejemplos: 
    
    \it{cigarro} ha pasado al francés como \it{cigare}, al italiano como \it{sigaro} o al inglés como \it{sigar}.
    
    \item Por razones de prestigio evidentes, muchas lenguas han tomado préstamos provenientes del inglés.
    
    Ejemplos: 
    
    Las palabras japonesas \it{suupaa}/\it{suupaamaaketto}, \it{taipuraitaa} o \it{boyifurendo}. 
\end{itemize}



\end{frame}


\begin{frame}{Calco o préstamo en traducción}

\begin{itemize}
    \item Un tipo especial de préstamo es el \textcolor{blue}{calco}, que consiste en una traducción literal de los elementos de una palabra de una determinada lengua a la lengua que la toma prestada.

    Ejemplos: 
    
    \it{skyscraper} ha pasado como calco al castellano (\it{rascacielos}), al francés (\it{gratteciel}), al holandés (\it{wolkenkrabber}) o al alemán (\it{Wolkenkratzer}).
    
    \item En castellano, \it{superhombre} es un calco que ha llegado a través del inglés \it{superman}, que a su vez lo tomó como calco del alemán \it{Übermensch}
    
\end{itemize}

\end{frame}

\begin{frame}{Composición}

\begin{itemize}
	\item La \textcolor{blue}{composición} es el proceso mediante el cual se combinan dos o más palabras para dar origen a una tercera que es nueva.
	
	Ejemplos: 
	
	\it{sacacorchos, lavaplatos, abrelatas, quitanieves, lanzallamas, guardarropa, pisapapeles}
	
	\item Aunque la mayor parte de las palabras que se utilizan en los compuestos son verbos y sustantivos que originan sustantivos, también se puede encontrar ejemplos de composición con adjetivos: \it{albiceleste, rojiverde, rojiblanco}
	
	\item La composición es un recurso muy productivo en lenguas como el alemán y el inglés.
\end{itemize}

\end{frame}


\begin{frame}{Mezcla}

\begin{itemize}
	\item La \textcolor{blue}{mezcla} es otro el proceso de creación de nuevas palabras a partir de la combinación de dos formas independientes.
	\item A diferencia de lo que ocurre en la composición, en la mezcla se combinan partes de palabras y no palabras completas.
	
	Ejemplos: 
	
	\it{smog} como mezcla de \it{smoke} + \it{fog}

	\it{brunch} como mezcla de \it{breakfast} + \it{lunch}.

	\it{motel} como mezcla de \it{motor} + \it{hotel}.

	\it{Chunnel} como mezcla de \it{tunnel} + \it{Channel}.

	\it{infotainment} como mezcla de \it{information} + \it{entertaiment}.

	\it{Spanglish} como mezcla de \it{Spanish} + \it{English}
	
	\it{telethon} como mezcla de \it{television} + \it{marathon}
	
\end{itemize}

\end{frame}

\begin{frame}{Apócope}

\begin{itemize}
	\item El \textcolor{blue}{apócope} consiste en la reducción de una palabra formada por más de una sílaba para formar una palabra más corta.
	
	Ejemplos: 
	
	\it{fax, foto, fan, moto, bici}, etc.
		
	\item El entorno escolar invita a acortar palabras y encontramos múltiples ejemplos de apócope.
	
	Ejemplos: 
	
	\it{cole, profe, boli, mates, natu, cono}, etc.
\end{itemize}

\end{frame}

\begin{frame}{Hipocorísticos}
	
\begin{itemize}
	\item Un tipo especial de reducción que da lugar a formas cariñosas o familiares son los \textcolor{blue}{hipocorísticos}: 
	
	\item Su formación es muy común en inglés británico y australiano, donde se forman recortando las palabras a una sílaba y añadiendo las terminaciones -y o -ie.

	Ejemplos: 
	
	\it{movie} proviene de \it{moving pictures}, \it{telly} proviene de \it{television}, \it{barbie} proviene de \it{barbecue}, \it{aussie} proviene de \it{Australian}, \it{brekky} proviene de \it{breakfast}, \it{bookie} proviene de \it{book maker}, etc.
	
	\item Otros ejemplos típicos de hipocorísticos son las reducciones en los nombres propios de persona: \it{Toni, Cris, Eli, Mari, Juli}, etc.
\end{itemize}

\end{frame}

\begin{frame}{Retroformación}

\begin{itemize}
	\item La \textcolor{blue}{retroformación} es un tipo muy especializado de reducción que consiste en una palabra de una determinada clase (normalmente, un nombre) se reduce para dar lugar a una palabra de otra clase diferente (normalmente, un verbo).
	
	Ejemplos:
	
	\it{televisión} dio lugar a \it{televisar}
	
	\it{donation} dio lugar a \it{donate}

	\it{option} dio lugar a \it{opt}

	\it{babysitter} dio lugar a \it{babysit}

	\it{enthusiasm} dio lugar a \it{enthuse}

	\it{editor} dio lugar a \it{edit}
\end{itemize}

\end{frame}

\begin{frame}{Conversión}

\begin{itemize}
	\item La \textcolor{blue}{conversión} es el proceso mediante el cual una palabra de una determinada clase comienza a utilizarse como una palabra de otra clase diferente, sin que haya habido ningún tipo de reducción.
	
	\item Este proceso se conoce también con la denominaciones de \textcolor{blue}{cambio de categoría} o \textcolor{blue}{cambio funcional}.

	Ejemplos:
	
	\it{chair, vacation}, originariamente sustantivo, se puede utilizar como verbo:  \it{Who's chairing the meeting?}, \it{I'm vacationing in Ibiza}.
	
	\it{must, guess}, originariamente verbos, se pueden utilizar como sustantivos:  \it{This book is a must}.

	\it{print out, take over, want to be}, originariamente verbos, se pueden utilizar como sustantivos:  \it{He's just a wannabe}.
	\item En castellano, el único ejemplo similar es el verbo \it{concienciar(se)}, formado por conversión a partir de \it{conciencia}.
\end{itemize}
\end{frame}


\begin{frame}{Acrónimos}

\begin{itemize}
	\item Algunas palabras nuevas conocidas como \textcolor{blue}{acrónimos} se forman a partir de las siglas o iniciales de un grupo de palabras: \it{CD, VCR} provienen de \it{compact disc} y \it{video cassette recorder}

	\item En algunos acrónimos se opta por pronunciarlos como palabras únicas: \it{OTAN, UNED, RENFE}.
	
	\item Otros acrónimos son tan habituales que incluso se escriben con letras minúsculas: \it{láser} proviene de \it{light amplification by stimulated emision of radiation}, \it{radar} proviene de \it{radio detecting and ranging}, \it{pin} proviene de \it{personal identification number}.
\end{itemize}
\end{frame}

\begin{frame}{Derivación}

\begin{itemize}
	\item El mecanismo de formación de nuevas palabras más común es la \textcolor{blue}{derivación}.
	\item Consiste en la formación de nuevas palabras haciendo uso de pequeñas «piezas» de la lengua que no figuran en los diccionarios como palabras independientes: los \textcolor{blue}{afijos}.
	
	Ejemplos:
	
	los elementos \it{anti-, des-, pre-, -ción, -ero} que aparecen en ejemplos de palabras derivadas como \it{antisocial, desagradecido, prejuzgar, luxación, cartero}.
	\end{itemize}
\end{frame}

\begin{frame}{Prefijos y sufijos}

\begin{itemize}
	\item Los afijos que se añaden a principio de la palabra se denominan \textcolor{blue}{prefijos}.
	\item Otro tipo de afijos se añaden al final de la palabra y se denominan \textcolor{blue}{sufijos}.
	\item Todas las palabras generadas mediante un proceso de derivación cuentan con prefijos, sufijos o con ambos tipos de afijos:
	
	Ejemplos:
	
	\it{des-prender} 

	\it{region-al} 

	\it{des-nacion-al-izar} 
\end{itemize}
\end{frame}

\begin{frame}{Infijos}

\begin{itemize}
	\item Existe un tercer tipo de afijos que aparecen en mitad de la palabra y se denominan \textcolor{blue}{infijos}.
	\item Son muy comunes en algunas lenguas, aunque en castellano su uso es limitado. Existen algunos ejemplos: \it{-ad-} en \it{pan\textbf{ad}ero}, \it{-ic-} en \it{carn\textbf{ic}ero}, \it{-l-} en \it{cafe\textbf{l}ito}.
	\item En inglés, los infijos tampoco son comunes: \it{Absofuckinglutly}, \it{Hallebloodylujah}.
	\item Otras lenguas como el kamhmu proporcionan mejores ejemplos:
	
	\it{see} (taladrar) / \it{s\textbf{rn}ee} (un taladro) 
	
	\it{toh} (cincelar) / \it{t\textbf{rn}oh} (un cincel) 
	
	\it{hiip} (comer con cuchara) / \it{h\textbf{rn}iip} (una cuchara) 
\end{itemize}
\end{frame}

\begin{frame}{Procesos múltiples en la creación de palabras}
	
\begin{itemize}
	\item Es común encontrar más evidencias de que más de unos de los procesos explicados anteriormente actúan de forma simultánea en la creación de una palabra.
	
	\item \it{deli} en inglés amerciano fue originariamente un préstamo del alemán \it{delicatessen} que se redujo por apócope

	\item En la frase \it{Problems have snowballed}, \it{snowball} era originariamente una palabra compuesta que sufrió una conversión para utilizarse como verbo.

\end{itemize}
\end{frame}


\begin{frame}{Parte 2}

\begin{center}
  \LARGE{Morfología: las palabras y los morfemas}
\end{center} 

\end{frame}


\begin{frame}{La palabra}

\begin{itemize}
	\item La palabra `palabra' puede parecer precisa para las necesidades ordinarias, pero resulta tremendamente ambigua cuando tratamos de analizar palabras a distintos niveles lingüísticos.
	\item A un nivel general, podemos distinguir entre unidades genéricas abstractas (\textcolor{blue}{palabra-tipo}) o unidades de texto o de discurso (\textcolor{blue}{palabra-ocurrencia}).
	
	La oración \it{La casa de Pedro es más bonita que la casa de Juan} tiene 12 palabras-ocurrencia pero solo 9 palabras-tipo.
	\item Desde el punto de vista fonológico, la palabra \ipa{/póyo/} para un yeísta o \ipa{/sié}N\ipa{to/} para un hablante que sesee, representan dos palabras distintas cada una: \it{pollo/poyo} y \it{ciento/siento}. 
\end{itemize}
\end{frame}
	
\begin{frame}{La palabra}

\begin{itemize}
	\item En francés, la palabra fonológica \ipa{/S\=a"tE/} equivale a varias palabras ortográficas distintas: \it{chantais, chantait, chantaient, chanter, chanté, chantée}.
	\item A su vez, se corresponden con otras tantas palabras gramaticales: 1ª, 2ª y 3ª pers. sing. y 3ª pers. plur. del pret. imperf. de indicativo, pero también infinitivo, part. pasado, etc.
	\item A cierto nivel de abstracción, podemos entender estas palabras gramaticales como variantes de una misma unidad, un mismo \textcolor{blue}{lexema}, que identificamos como el verbo \it{chanter}. De hecho, es este representante canónico el que suele aparecer como entrada en los diccionarios.
	
\end{itemize}

\end{frame}

\begin{frame}{La palabra}

Por lo tanto, podemos distinguir diferentes sentidos para la palabra `palabra':

\begin{itemize}
	\item palabra-tipo o palabra-ocurrencia
	\item palabra fonológica u ortográfica
	\item palabra gramatical, representada por la palabra ortográfica o fonológica
	\item lexema, o unidad abstracta que se representa bajo diferentes formas fonológicas u ortográficas
\end{itemize}
	

\end{frame}

\begin{frame}{La palabra}

\begin{itemize}
	\item En swahili, la idea de palabra difiere bastante de la que tenemos en castellano, ya que una palabra como \it{nitakupenda} se correponde con varias palabras en nuestra lengua `yo te querré'. 
\end{itemize}

\begin{center}
\begin{tabular}{l l l l l}
swahili: ni & -ta-  & -ku- & -penda\\
& & & & \\
español: yo & -rré & te & querer\\
\end{tabular}
\end{center}


\end{frame}

\begin{frame}{Morfemas}

\begin{itemize}
	\item No necesitamos recurrir a lenguas distantes para descubir que las palabras está formadas por varios elementos.
	\item En las palabas \it{cantas, cantante, cantaba, cantando} podemos identificar un elemento constante \it{cant-} y distintos elementos como \it{-as, -ante, -aba, -ando}.
	\item Todos estos elementos son \textcolor{blue}{morfemas}.
	\item El \textcolor{blue}{morfema} se define como ``la unidad mínima de significado o función gramatical''. 
	\item La función gramatical es la que nos permite indicar el plural o el tiempo presente.
\end{itemize}

\end{frame}

\begin{frame}{Morfemas}

\begin{itemize}
	\item En la palabra \it{reabrirá} de la oración \it{La policía reabrirá el caso} podemos identificar tres morfemas.
	
	Una unidad mínima de significado \it{abrir}, otra unidad mínima de significado \it{re-} que significa ``de nuevo'' y otra unidad mínina con función gramatical \it{-rá} que indica tercera persona del singular y tiempo futuro. 
	
	\item En la palabra \it{paneras} también podemos identificar tres morfemas distintos.
	
	Una unidad mínima de significado \it{pan}, otra unidad mínima de significado \it{-era} que significa ``relativo a'' y una tercera unidad con función gramatical \it{-s} que indica número plural.
\end{itemize}

\end{frame}

\begin{frame}{La morfología}

La morfología se ocupa de las unidades gramaticales dotadas de significado o morfemas. Como disciplina, se ocupa principalmente de:

\begin{itemize}
	\item determinar cuáles son las unidades morfológicas.
	\item estudiar las distintas variaciones que adoptan estas unidades.
	\item estudiar la disposición de las unidades morfológicas para formar unidades superiores como la palabra.
	\item representar el conocimiento que un hablante tiene de cómo se organizan los morfemas para formar palabras
\end{itemize}

\end{frame}


\begin{frame}{Morfemas libres y morfemas ligados}

A partir de los ejemplos anteriores, podemos distinguir dos tipos principales de morfemas:

\begin{itemize}
	\item Los \textcolor{blue}{morfemas libres} son aquellos que pueden aparecer como palabras independientes, como \it{pan}. 
	
	Podemos considerar a los morfemas libres como el conjunto de las palabras individuales del castellano: los sustantivos, adjetivos, verbos, etc. básicos. 
	\item Los \textcolor{blue}{morfemas ligados} son aquellos que no pueden aparecer de forma independiente y necesitan concatenarse con otros elementos, \it{re-, -era, -s}. 
	
	Los morfemas ligados son los afijos (prefijos, sufijos e infijos).
\end{itemize}

\end{frame}

\begin{frame}{Morfemas léxicos y morfemas funcionales}

Podemos dividir los morfemas libres en dos tipos:	

\begin{itemize}
	\item Los \textcolor{blue}{morfemas léxicos} son los nombres, adjetivos y verbos comunes que consideramos habitualmente como portadores del ``contenido'' de los mensajes.
	
	Como ejemplos podemos mencionar \it{chico, árbol, azul, feliz, abrir, romper}.
	
	Dado que siempre podemos incluir nuevos morfemas léxicos a una lengua, decimos que los morfemas léxicos constituyen una clase ``abierta'' de palabras. 
\end{itemize}
\end{frame}

%-% TODO: explicar qué son las raíces 

\begin{frame}{Morfemas léxicos y morfemas funcionales}

\begin{itemize}
	\item Los \textcolor{blue}{morfemas funcionales o gramaticales} son las palabras funcionales de la lengua como preposiciones, artículos, conjunciones y pronombres.
	
	Como ejemplos podemos mencionar \it{contra, cerca, cuando, el, la, que, y}.
	
	Dedibo a que nunca se pueden incluir nuevos morfemas gramaticales a una lengua, decimos que éstos constituyen una clase ``cerrada'' de palabras. 
	
\end{itemize}
\end{frame}

\begin{frame}{Morfemas derivativos y morfemas flexivos}

El conjunto de los afijos o morfemas ligados puede dividirse en dos grupos:
\begin{itemize}
	\item Los \textcolor{blue}{morfemas derivativos} se utilizan para formar nuevas palabras o para generar palabras de una categorías diferente a la de la raíz.
	
	Al añadir el morfema derivativo \it{-ción} a un verbo como \it{donar} obtenemos el sustantivo \it{donación}. 
	
	De un modo similar, el morfema derivativo \it{-oso} se combina con el sustantivo \it{calor} para formar el adjetivo \it{caluroso}.
	
	Una lista de morfemas derivativos en castellano incluiría sufijos como \it{-al, -mente, -miento} o prefijos como \it{re-, pre-, ex-, dis-, in-}. 
\end{itemize}

\end{frame}

\begin{frame}{Morfemas derivativos y morfemas flexivos}

\begin{itemize}
	\item Los \textcolor{blue}{morfemas flexivos} no se utilizan generalmente para formar nuevas palabras, sino para indicar diversos aspectos de la función gramatical de una palabra: número (singular o plural), género (masculino o femenino), tiempo (presente, pasado o futuro), o si se trata de una forma superlativa.
	
	En inglés sólo existen ocho morfemas flexivos, pero el castellano es una lengua mucho más rica en este sentido.
	
    Ej.:
    
    Quier\textcolor{blue}{o} dec\textcolor{blue}{ir} alg\textcolor{blue}{o} sobre l\textcolor{blue}{as} herman\textcolor{blue}{as} de Montse.
    
    Reyes siempre est\textcolor{blue}{aba} cant\textcolor{blue}{ando}.

	Beatriz \textcolor{blue}{era} inteligent\textcolor{blue}{ísima}: prefer\textcolor{blue}{ía} estudi\textcolor{blue}{ar} y siempre le\textcolor{blue}{ía}.	
\end{itemize}

\end{frame}

\begin{frame}{Morfemas derivativos y morfemas flexivos}

A partir de los ejemplos anteriores, podemos deducir que hay determinados morfemas flexivos que se combinan con:

\begin{itemize}
	\item sustantivos para marcar el género (\it{-o, -a}).
	\item verbos para marcar la persona y el tiempo (\it{-o, -ir, -aba, -ndo -ía}).
	\item adjetivos para marcar el grado superlativo (\it{-ísima}).
	\item El sufijo de número (\it{-s}) parece ser común.
\end{itemize}
	
Es preciso recordar que los morfemas flexivos presentan mucha variabilidad. P. ej., la marca de plural puede indicarse con \it{-s, -es} y la tercera persona del singular en presente puede ser \it{-a, -e}.
\end{frame}

\begin{frame}{Morfemas derivativos y morfemas flexivos}

Observamos distintas diferencias importantes entre morfemas derivativos y flexivos:

\begin{itemize}
	\item Un morfema flexivo nunca cambia la categoría de la palabra.
	
	En inglés, el adjetivo \it{old} combinado con el morfema \it{-er} simplemente forma una nueva versión del adjetivo: \it{older}.
	 
	\item Un morfema derivativo puede cambiar la categoría de la palabra. 
	
	En inglés, el verbo \it{teach} combinado con el morfema \it{-er} forma un nuevo sustantivo \it{teacher}.
	
	%El hecho de que ambos morfemas tengan la misma forma \it{-er} no quiere decir que desempeñen la misma función. De hecho, cada uno proviene de una forma distinta del inglés antiguo, pero han evolucionado a la misma forma. 

	\item Cuando una misma palabra contiene tanto morfemas derivativos como flexivos siempre aparecen en este orden.
\end{itemize}
	
\end{frame}

\begin{frame}{Tipos de morfemas}

\Tree [.morfemas [.libres [.léxicos ] [.funcionales ]] [.ligados [.derivativos ] [.flexivos ]]]

\end{frame}


\begin{frame}{Descripción morfológica}

La oración \it{La buenísima actuación de la corredora emocionó a sus entrenadores} contiene 21 morfemas.

\vspace{0.3cm}

\begin{center}
\begin{tabular}{l l l l l}
\it{La} & \it{buen-} & \it{-ísim-} & \it{-a} & \it{actua-} \\
funcional & léxico & flexivo & flexivo & léxico \\
& & & & \\
\it{ción} & \it{de} & \it{la} & \it{corr-} & \it{-ed-} \\
derivativo & funcional & funcional & léxico & derivativo \\
& & & & \\
\it{-or} & \it{-a} & \it{emocion-} & \it{-ó} & \it{a} \\
derivativo & flexivo & léxico & flexivo & funcional \\
& & & & \\
\it{su-} & \it{-s} & \it{entren} & \it{-ad-} & \it{-or-} \\
funcional & flexivo & léxico & derivativo & derivativo \\
& & & & \\
\it{-es} & & & &  \\
flexivo & & & &  \\
\end{tabular}
\end{center}

\end{frame}


\begin{frame}{Problemas de la descripción morfológica}

\begin{itemize}
	\item Hasta ahora, solo hemos considerado ejemplos de palabras en las que los distintos morfemas se pueden identificar fácilmente como elementos discretos.
	\item Existen problemas significativos en el análisis de la morfología en castellano.
	\item ¿Cuál es el morfema flexivo que indica el plural en castellano de \it{crisis} o hace que \it{men} sea el plural de \it{man} en inglés?
	\item ¿Cuál es el morfema flexivo que hace de la palabra \it{fui} el pasado el verbo \it{ir}?
	\item Si \it{-al} es un morfema derivativo que podemos combinar con sustantivos como \it{nación} para formar \it{nacional}, ¿significa eso que la palabra \it{filial} está formado por los morfemas \it{fili- + -al}? 
\end{itemize}

\end{frame}

\begin{frame}{Morfos y alomorfos}

\begin{itemize}
	\item Una posible solución para analizar las diferencias entre los distintos morfemas flexivos consiste en proponer variaciones en las reglas de formación morfológica, adoptando la analogía que hemos visto en fonología.
	\item Del mismo modo que teníamos \textcolor{blue}{fonos}, que eran la realización fonética real de los fonemas, ahora podemos hablar de \textcolor{blue}{morfos} y definirlos como las formas utilizadas realmente a la hora de generar morfemas.
	\item Del mismo modo que en fonología hablábamos de alófonos como las distintas realizaciones fonética de un mismo fonema, en morfología podemos hablar de \textcolor{blue}{alomorfos} como las distintas versiones de un mismo morfema.
\end{itemize}

\end{frame}


\begin{frame}{Morfos y alomorfos}

En castellano, el morfema que indica el plural tiene tres alomorfos.
	
\begin{itemize}
	\item \it{casas}: \it{casa} + plural = \it{-s}
	\item \it{camiones}: \it{camion} + plural = \it{-es}
	\item \it{crisis}: \it{crisis} + plural = $\varnothing$
\end{itemize}

En la primera persona del pretérito perfecto simple del castellano encontramos los siguiente alomorfos: 

\begin{itemize}
	\item \it{amé}: \it{amar} + 1ª pers. sing pps = \it{-é}
	\item \it{temí}: \it{temer} + 1ª pers. sing pps = \it{-í}
	\item \it{fui}: \it{ir} + 1ª pers. sing pps = \it{fuí}
\end{itemize}

En inglés, en el ejemplo \it{man/men} se ha producido un cambio vocálico \ipa{/\ae/} $\rightarrow$ \ipa{/E/} que se correspondería con un morfo que da lugar a este ejemplo de plural irregular.

\end{frame}

\begin{frame}{El morfema $\varnothing$ (nulo)}

\begin{itemize}
	\item El morfema de tiempo en castellano puede indicarse como \it{-ba, -ra, -ía}. El morfema de persona, se indica como \it{-s}. Pero puede darse el caso en el que el morfema no se manifieste: el presente \it{cant+$\varnothing$+a}.
	\item Si el morfema que aparece como $\varnothing$ nunca se manifiesta de otra forma, entonces no es legítimo proponerlo como morfema $\varnothing$. 
	
	El singular en castellano no tiene representante morfológico, por lo tanto no puede hablarse de morfema $\varnothing$. 
	
	El morfema de género en castellano se suele manifestar como (\it{moz-o, moz-a}), pero también existe en la forma de morfema $\varnothing$: \it{señor+$\varnothing$}  
	\item Este morfema $\varnothing$ es la forma de decir que un cierto morfema no tiene contenido expreso. 
\end{itemize}
	

\end{frame}

\begin{frame}{Análisis morfológicos en otras lenguas}

Cuando analizamos morfológicamente otras lenguas podemos encontrar formas y patrones que podríamos explicar a patir de las categorías básicas de morfemas que hemos visto para el castellano.

\begin{center}
\begin{tabular}{l l l l}
\it{dark-} `oscuro' & \it{-en-} `hacer' & \it{-ed} `pasado' & \it{darkened} `oscureció' \\
léxico & derivativo & flexivo & \\
& & & \\
\it{mic-} `morir' & \it{-tia-} `causar' & \it{-s} `futuro' & \it{mictias} `matará' \\
léxico & derivativo & flexivo & \\
& & & \\
\end{tabular}
\end{center}

\end{frame}

\begin{frame}{Análisis morfológicos en otras lenguas}

El canurí es una lengua hablada en Nigeria.

\begin{center}
\begin{tabular}{l l}
Adjetivo & Sustantivo \\
\it{karite} `excelente' & \it{n\ipa{@}mkarite} `excelencia'\\
\it{kura} `grande' & \it{n\ipa{@}mkura} `grandeza'\\
\it{gana} `pequeño' & \it{n\ipa{@}mgana} `pequeñez'\\
\it{dibi} `malo' & \it{n\ipa{@}mdibi} `maldad'\\
\end{tabular}
\end{center}

En conclusión, si en canurí \it{n\ipa{@}mkurugu} significa `longitud', `largo' se dirá \pause \it{kurugu}. \\

\end{frame}

\begin{frame}{Análisis morfológicos en otras lenguas}

El ganda es una lengua hablada en Uganda.

\begin{center}
\begin{tabular}{l l}
Singular & Plural \\
\it{omusawo} `doctor' & \it{abasawo} `doctores'\\
\it{omukazi} `mujer' & \it{abakazi} `mujeres'\\
\it{omuwala} `chica' & \it{abawala} `chicas'\\
\it{omusika} `heredero' & \it{abasika} `herederos'\\
\end{tabular}
\end{center}

En conclusión, si en ganda \it{abalenzi} significa `chicos', `chico' se dirá \pause \it{omulenzi}. \\

\end{frame}

\begin{frame}{Análisis morfológicos en otras lenguas}

El ilocano es una lengua hablada en Filipinas que presenta una forma completamente diferente de marcar el plural a través de un proceso conocido con el nombre de reduplicación.

\begin{center}
\begin{tabular}{l l}
Singular & Plural \\
\it{úlo} `cabeza' & \it{ulúlo} `cabezas'\\
\it{dálan} `camino' & \it{daldálan} `caminos'\\
\it{bíag} `vida' & \it{bibíag} `vidas'\\
\it{múla} `planta' & \it{mulmúla} `plantas'\\
\end{tabular}
\end{center}

En conclusión, si en ilocano \it{taltálon} significa `campos', `campo' se dirá \pause \it{tálon}. \\

\end{frame}

\begin{frame}{Análisis morfológicos en otras lenguas}

El tagalo es una de las principales lenguas habladas en Filipinas.

\begin{center}
\begin{tabular}{l l l}
Singular & Plural \\
\it{basa} `leer' & \it{tawang} `llamar' & \it{sulat} `escribir' \\
\it{bumasa} `¡lee!' & \it{tumawag} `¡llama!' & \it{sumulat} `¡escribe!' \\
\it{babasa} `leerá' & \it{tatawag} `llamará' & \it{susulat} `escribirá' \\
\end{tabular}
\end{center}	

En conclusión, si en tagalo \it{lapit} significa `venir aquí', `ven aquí' y `vendrá aquí' se dirán, respectivamente \pause \it{lumapit} y \it{lalapit}. 

Y si \it{lalakad} significa `andará', ¿cómo será la forma para `andar? \pause Solo puede ser \it{lakad}.

\end{frame}

\begin{frame}{Resumen}

\begin{itemize}
	\item La palabra es una unidad compleja cuya estructura interna es el objeto de la morfología.
	\item Las unidades de la morfología son los morfemas.
	\item El morfema es la unidad lingüística más pequeña provista de significado.
	\item Un alomorfo es cada una de las posible realizaciones (una variante distribucional) de un mismo morfema.
	\item La palabra está formada habitualmente por un morfema léxico y uno o varios afijos (morfemas derivativos o flexivos) o un morfema funcional.
	\item Los morfemas derivativos modifican generalmente la categoría gramatical de la unidad con la que se combinan. 
\end{itemize}
	
\end{frame}

\begin{frame}{Ejercios: morfema flexivo del plural en turco}
	
\begin{center}
\begin{tabular}{l l l l}
& Singular & Plural & \\
`hombre' & \it{adam} & \it{adamlar} & `hombres' \\
`arma' &  & \it{toplar} & `armas' \\
`lección' & \it{ders} &  & `lecciones' \\
`lugar' & \it{yer} & \it{yerler} & `lugares' \\
`camino' &  & \it{yollar} & `caminos' \\
`cerrojo' & & \it{kilitler} & `cerrojos' \\
`flecha' & \it{ok} &  & `flechas' \\
`mano' & \it{el} &  & `manos' \\
`brazo' & \it{kol} &  & `brazos' \\
`campana' &  & \it{ziller} & `campanas' \\
`amigo' &  & \it{dostlar} & `amigos' \\
`manzana' & \it{elma} &  & `manzanas' \\
\end{tabular}
\end{center}	

\end{frame}



\begin{frame}{Parte 3}

\begin{center}
  \LARGE{Clases de palabras: la gramática tradicional}
\end{center} 

\end{frame}

\begin{frame}{Análisis lingüístico}

Hasta ahora hemos visto que podemos describir la lengua a distintos niveles: \it{los alumnos estresados}.

\begin{itemize}
	\item Fonológico: \ipa{/los alú}N\ipa{nos estresádos/}
	
	\begin{itemize}
		\item \ipa{/u/} es un fonema vocálico posterior cerrado
		\item \ipa{/t/} es un fonema consonántico oclusivo dental sordo
		\item /N/ es el archifonema resultado de neutralizar una consonante nasal en posición de sílaba trabada
	\end{itemize}
\end{itemize}

\end{frame}


\begin{frame}{Análisis lingüístico}

\begin{itemize}
	\item Morfógico
\end{itemize}
	
\begin{center}
\begin{tabular}{l l l l l l}
l- & -o- & -s & alumn- & -o & -s \\ 
func & flex & flex & léx & flex & flex \\
 & & & & & \\
estres- & -ad- & -o & -s & & \\
léx & derivat & flex & flex & & \\
\end{tabular}
\end{center}	

Mediante este tipo de descripciones podemos caracterizar todas las palabras y sintagmas de una lengua, según su fonología y su morfología.
	
\end{frame}

\begin{frame}{Gramática}

\begin{itemize}
	\item Sin embargo, hasta ahora no hemos hablado de cuáles son las reglas que explican las posibilidades combinatorias de estas palabras. 
	\item Un sintagma como \it{los alumnos estresados} es una expresión correctamente construida y válida en castellano. Es gramatical.
	\item Por el contrario, expresiones como la siguientes no son correctas. Son agramaticales. 
	
	* \it{estresados alumnos los}
	
	* \it{alumnos los estresados}
	
	\item La \textcolor{blue}{gramática} puede definirse como un conjunto de reglas  que permiten describir la estructura de los sintagmas y de las oraciones, de manera que podamos caracterizar todas las construcciones gramaticales y distinguirlas de las agramaticales. 
\end{itemize}
\end{frame}

\begin{frame}{Las partes de la oración en la gramática tradicional}

\begin{itemize}
    \item Los términos como `nombre', `verbo' o `adjetivo' que utilizamos habitualmente para etiqueta las distintas clases de palabras provienen de una tradición gramatical cuyos orígenes se remontan al estudio y a la descripción de lenguas como el griego y el latín.
    \item En su momento cuando el análisis lingüístico se amplió a otras lenguas, lo más apropiado resultó adoptar sin más este tipo de clasificación. 
    \item Otras categorías como el el número, la persona, el género, el tiempo, la voz o el modo provienen de esta misma gramática tradicional.
\end{itemize}

\end{frame}


\begin{frame}{Las partes de la oración en la gramática tradicional}

\begin{center}
\begin{tabular}{l l l l l l l}
Los & chicos, & ya & cansados, & veían & a & los \\
DET & N & ADV & ADJ & VB & PREP & DET \\
& & & & & & \\
payasos & y & aplaudían & con & poca & fuerza. & \\
N & CONJ & VB & PREP & ADJ & N & \\
\end{tabular}
\end{center}	

\end{frame}

\begin{frame}{Las partes de la oración en la gramática tradicional}

\begin{itemize}
    \item nombres (N): palabras que utilizamos para designar personas (\it{chicos}), cosas (\it{coches}), seres vivos (\it{gata}), lugares (\it{mar}), cualidades (\it{color}), fenómenos (\it{tormenta}) e ideas abstractas (\it{libertad}) como si fueran «cosas». 
    \item determinantes (DET): palabras que acompañan a los nombres (\it{el, una, estos}) y se emplean para construir sintagmas nominales que permitan clasificar las cosas denotadas por dichos nombres (\it{un tomate}) o identificar las cosas como objetos ya conocidos (\it{el tomate}).   
\end{itemize}

\end{frame}


\begin{frame}{Las partes de la oración en la gramática tradicional}

\begin{itemize}
    \item adjetivos (ADJ): palabras que acompañan a los nombres aportando más información sobre las cosas denotadas por dichos nombres (\it{niños maleducados, buena música}). 
    \item verbos (V): palabras que utilizamos para referirnos a acciones (\it{caminar, saltar}) o  estados (\it{ser, estar}) que involucran a personas o cosas en diversos sucesos.
    \item adverbios (ADV): palabras que proporcionan información adicional sobre las acciones o estados denotados por los verbos (\it{pronto, rápidamente, ayer}). Algunos adverbios también se utilizan como modificadores de adjetivos (\it{una historia muy extraña, la cena está casi lista}).
\end{itemize}
\end{frame}

\begin{frame}{Las partes de la oración en la gramática tradicional}

\begin{itemize}
    \item preposiciones (PREP): palabras que se emplean junto a los nombres o a los sintagmas nominales para construir sintagmas preposicionales que proporcionan información temporal (\it{a las diez, por la noche}), de lugar (\it{en mi casa, sobre la mesa}), de instrumento (\it{con un martillo}), de causa (\it{por imprudente})
    \item conjunciones (CONJ): son palabras utilizadas como conectores para indicar relaciones entre sucesos.    
    \item pronombres (PRON): palabras que se utilizan en lugar de los nombres o los sintagmas nominales y que, normalmente, se refieren a personas u objetos previamente conocidos (\it{ella me dijo que éste era para mí}).
\end{itemize}
\end{frame}


\begin{frame}{La concordancia}

\begin{itemize}
    \item Categorías como el número, la persona, el tiempo, el género pueden tratarse por separado, pero resulta más sencillo explicarlas centrándonos en el fenómeno de la \textcolor{blue}{concordancia}.
    \item En una frase como \it{La niña pasea a su perro}, decimos que el verbo \it{pasea} concuerda con el nombre \it{niña}.
    \item La concordancia en castellano se materializa en que el verbo concuerda en persona --distinguiendo entre el hablante, el oyente o el resto de los sujetos-- y número --distinguiendo entre singular y plural-- con el sujeto.
    \item Las diferentes formas de los pronombres también se describen en términos de persona y número, p. ej. en \it{Ella pasea a su perro}. 
\end{itemize}
\end{frame}


\begin{frame}{La concordancia}

La forma del verbo se describe también en términos de otras categorías: el tiempo, la voz y el modo.

\begin{itemize}
    \item El tiempo es la categoría gramatical que indica si la acción se realiza en el momento actual, en el pasado o en el futuro.     
    \item El modo es la categoría gramatical que clasifica la acción, el proceso o el estado de un verbo desde la perspectiva del emisor, según éste la conciba como real (indicativo), subjetiva (subjuntivo) o apelativa (imperativo).
    \item La voz es la categoría gramatica que indica quién realiza (agente) y quién recibe (paciente) la acción denotada por el verbo, es decir, indica el papel temático del sujeto y del objeto. \it{El perro es paseado}
\end{itemize}
\end{frame}


\begin{frame}{La concordancia}

\begin{itemize}
    \item Por último, la categoría de género permite describir concordancia entre \it{perro} y \it{paseado}.
    \item El género es una propiedad lingüística y no siempre hay relación directa y lógica con respecto al sexo biológico. 
    \item En inglés o en castellano, es normal que exista relación directa entre género gramátical y género natural. 
    
    \it{The girl walks her dog} / \it{La niña pasea a su perro}.
\end{itemize}
\end{frame}



\begin{frame}{El género gramatical}

\begin{itemize}
    \item Al contrario de los que hemos visto en los ejemplos anteriores de inglés y castellano, en algunas lenguas se emplea el llamado \textcolor{blue}{género gramatical}.
    \item En estas lenguas, los sustantivos se agrupan de acuerdo con una clase de género (masculino o femenino) que no tiene por qué estar directamente relacionada con el sexo.
    \item En alemán existen tres géneros; masculino (\it{der Mond} `la luna'), femenino (\it{die Sonne} `el sol') y neutro (\it{das Feuer} `el fuego').  
    \item En alemán, el sintagma \it{das Mädchen} (`la chica joven') hace referencia,  biológicamente hablando, a una hembra. Sin embargo, el sintagma es gramaticalmente neutro.
\end{itemize}
\end{frame}


\begin{frame}{El análisis tradicional}

Conjugación del presente, modo indicativo, voz activa
\begin{center}
\begin{tabular}{l l l}
 & \bf{to love} & \bf{amare} \\
1 p. sing. & \it{I love} & \it{amo} \\ 
2 p. sing. & \it{you love} & \it{amas} \\ 
3 p. sing. & \it{he/she/it loves} & \it{amat} \\
\hline 
1 p. plur. & \it{we love} & \it{amamus} \\ 
2 p. plur. & \it{you love} & \it{amatis} \\ 
3 p. plur. & \it{they love} & \it{amant} \\ 
\end{tabular}
\end{center}

Este tipo de análisis es completamente lógico en latín, pero ¿tiene sentido una análisis como este en inglés?
\end{frame}


\begin{frame}{El enfoque prescriptivo}

\begin{itemize}
    \item Una cosa es adoptar terminología o nombres de categorías gramaticales de la tradición clásica y otra bien distinta es afirmar que la estructura de las oraciones inglesas debería ser similar a la de las oraciones latinas.
    \item Precisamente este fue el enfoque que plantearon los gramáticos del s. XVIII, los primeros en sistematizar la gramática y establecer el conjunto de reglas que definía qué usos eran correctos o convenientes y cuáles no.
    \item Esta «corrección lingüística» es una parte importante de la educación que recibimos. Se asume y acepta que alguien instruído que escribe bien respeta estas normas. Pero una persona que nos las respeta, es porque tiene bajo nivel cultural.
\end{itemize}

\end{frame}

\begin{frame}{El enfoque descriptivo}

\begin{itemize}
	\item El enfoque prescriptivo que se había utilizado con el latín se podía extender al estudio de lenguas romances, pero su uso era completamente contraproducente con lenguas alejadas.
	\item A finales del s. XIX, cuando se comenzó a estudiar lenguas americanas, los lingüistas terminaron adoptando un enfoque completamente diferente.
	\begin{enumerate}
		\item Se recogen datos y ejemplos en las lenguas que quieren estudiar.
		\item Se intenta describir las regularidades estructurales que se desprenden de esos datos.
		\item Dichas regularidades se utilizan para caracterizar los usos de la lengua.
	\end{enumerate}
\end{itemize}

\end{frame}


\begin{frame}{El análisis estructural}

\begin{itemize}
	\item Un tipo de enfoque prescriptivo se conoce con el nombre de \textcolor{blue}{análisis estructural}.
	\item El objetivo de este análisis es determinar la distribución de las distintas formas de una lengua.
	\item El método implica el uso de los llamados «marcos de prueba» que se suelen representar como oraciones con posiciones vacías.
	
	\it{La \_\_\_\_\_\_\_\_\_\_ hace mucho ruido.}

	\it{Oímos una \_\_\_\_\_\_\_\_\_\_ ayer.}

	\item Podemos asumir que todas las posibilidades que encajen en la posición vacía en un mismo «marco de prueba» pertenecen a una misma categoría gramatical: sustantivos.
\end{itemize}

\end{frame}

\begin{frame}{El análisis estructural}

\begin{itemize}
	\item Hay muchas otras opciones que no podrían encajar en los marcos de prueba anteriores y que darían lugar a oraciones agramaticales: \it{esto, un perro, el coche, el joven con bombín que se sienta al fondo}.
	
	*\it{La esto hace mucho ruido.}

	*\it{Oímos una un perro ayer.}

	\item Para estos ejemplos basta generar un marco de prueba distinto: 
	
	\it{\_\_\_\_\_\_\_\_\_\_ hace mucho ruido.}

	\it{Oímos \_\_\_\_\_\_\_\_\_\_ ayer.}

	%\item Del mismo modo, este marco de prueba nos permite generalizar y asumir que cualquier expresión que encaje en la posición vacía pertenecerá a una misma categoría gramatical.
	\item Podemos elaborar una descripción de las características estructurales de una lengua desarrollando un conjunto de marcos de prueba y determinando cuáles son las formas capaces de ocupar las posiciones vacías. 
\end{itemize}

\end{frame}


\begin{frame}{Análisis de constituyentes inmediatos}

\begin{itemize}
	\item Otro enfoque que tiene el mismo objetivo descriptivo es el conocido como \textcolor{blue}{análisis de constituyentes inmediatos}.
	\item La técnica se ha diseñado con objeto de mostrar cómo los constituyentes de menor tamaño se combinan para formar constituyentes mayores, cómo las palabras se unen para formar sintagmas.
	
	¿Cómo se agrupan los ocho constituyentes de la oración \it{Su padre llevó una pistola a la boda}?
\end{itemize}

\pause

\begin{center}
\begin{tabular}{|l|l|l|l|l|l|l|l|}
\hline 
\multicolumn{2}{|l|}{} & \multicolumn{3}{|l|}{} & \multicolumn{3}{|l|}{} \\
\cline{2-2}
\cline{4-5}
\cline{7-8}
Su & padre & llevó & una & pistola & a & la & boda\\ 
\hline 
\end{tabular}
\end{center}
\end{frame}


\begin{frame}{Análisis de constituyentes inmediatos}

\it{El perro enterraba un hueso}

\begin{center}
\begin{tabular}{|l|l|l|l|l|}
\hline 
\multicolumn{2}{|l|}{} & \multicolumn{3}{|l|}{} \\
\cline{2-2}
\cline{4-5}
El & perro & enterraba & un & hueso \\ 
\hline 
\end{tabular}
\end{center}

\it{La mujer escondió una gran serpiente en una caja}

\begin{center}
\begin{tabular}{|l|l|l|l|l|l|l|l|l|l|}
\hline 
\multicolumn{2}{|l|}{} & \multicolumn{4}{|l|}{} & \multicolumn{3}{|l|}{} \\
\cline{2-2}
\cline{4-6}
\cline{8-9}
La & mujer & escondió & una & gran & serpiente & en & una & caja \\ 
\hline 
\end{tabular}
\end{center}

\end{frame}

\begin{frame}{Análisis de constituyentes inmediatos}

\begin{itemize}
	\item Este tipo de diagramas se suele utilizar para mostrar qué elementos se pueden sustituir unos por otros en cada uno de los niveles de la estructura de constituyentes.
\end{itemize}

\begin{center}
\begin{tabular}{|l|l|l|l|l|l|l|l|l|l|}
\hline 
Su & padre & llevó & una & \multicolumn{2}{|l|}{pistola} & a & la & boda \\
\cline{5-6} 
La & mujer & escondió & una & gran & serpiente & en & una & caja \\ 
\cline{1-2} 
\cline{4-6} 
\cline{8-9} 
\multicolumn{2}{|l|}{Olga} & vio & \multicolumn{3}{|l|}{Friends} & en & \multicolumn{2}{|l|}{televisión} \\
\cline{7-9} 
\multicolumn{2}{|l|}{Ella} & abrió & \multicolumn{3}{|l|}{eso} & \multicolumn{3}{|l|}{cuidadosamente} \\
\hline 
\end{tabular}
\end{center}

\end{frame}

\begin{frame}{Oraciones etiquetadas y encorchetadas}

\begin{itemize}
	\item Otro tipo de diagrama común para mostrar de qué forma los distintos constituyentes de la estructura de la oración se combinan es mediante \textcolor{blue}{corchetes etiquetados}.
\end{itemize}

	\pause
	
	[El]_{Det} [perro]_{N} [enterraba]_{V} [un]_{Det} [hueso]_{N}

	\hspace{0.2cm}
	\pause
	
	[ [El]_{Det} [perro]_{N} ]_{SN} [enterraba]_{V} [ [un]_{Det} [hueso]_{N} ]_{SN}
	
	\hspace{0.2cm}
	\pause
	
	[ [El]_{Det} [perro]_{N} ]_{SN} [ [enterraba]_{V} [ [un]_{Det} [hueso]_{N} ]_{SN} ]_{SV}

	\hspace{0.2cm}
	\pause
	
	[ [ [El]_{Det} [perro]_{N} ]_{SN} [ [enterraba]_{V} [ [un]_{Det} [hueso]_{N} ]_{SN} ]_{SV} ]_{O}
\end{frame}


\begin{frame}{Oraciones etiquetadas y encorchetadas}

\begin{itemize}
	\item Al hacer este tipo de análisis no nos limitamos a etiquetar cada consituente con el nombre de su categoría gramatical, estamos haciendo explícita la organización jerárquica de la oración. 
	\item En esta jerarquía, la oración (O) es el nivel superior por encima de los sintagmas nominales (SN) y verbales (SV).
	\item Los sintagmas nominales están por encima de los nombres (N) y determinantes (Det).
	\item Los sintagmas verbales están compuesto por una forma verbal (V) y por un sintagma nominal.
	\item Podemos utilizar este tipo de diagramas para describir la estructura de los constituyentes de cualquier lengua, independientemente de la estructura de la oración. 
\end{itemize}
\end{frame}

\begin{frame}{Oraciones etiquetadas y encorchetadas en otras lenguas}

\begin{center}
\begin{tabular}{l l l l l l}
\it{Chunnaic} & \it{an} & \it{gille} & \it{an} & \it{cu} & \it{dubh} \\
`ver' & `el' & `niño' & `el' & `perro' & `negro' \\
\end{tabular}
\end{center}

\begin{center}
[Chunnaic]_{V} [an]_{Det} [gille]_{N} [an]_{Det} [cu]_{N} [dubh]_{Adj} 

\pause

[Chunnaic]_{V} [ [an]_{Det} [gille]_{N} ]_{SN} [ [an]_{Det} [cu]_{N} [dubh]_{Adj} ]_{SN} 

\pause

[ [Chunnaic]_{V} [ [an]_{Det} [gille]_{N} ]_{SN} [ [an]_{Det} [cu]_{N} [dubh]_{Adj} ]_{SN} ]_{O}
\end{center}

\begin{itemize}
	\item El diagrama muestra, en primer lugar, que el orden de constituyentes en gaélico es VSO, frente al orden SVO típico en castellano.
\end{itemize}

\end{frame}

\begin{frame}{Ejercicio}

	Traduce al latín la oración \it{Las palomas aman a la niña pequeña}:
	
	\vspace{0.5cm}
	
	\it{puellae aquilas portant}: `las niñas llevan las águilas'
	
	\it{feminae columbas amant}: `las mujeres aman las palomas'
	
	\it{puella aquilam salvat}: `la niña salva al águila'
	
	\it{femina paruam aquilam liberat}: `la mujer libera al águila pequeña'
	
	\it{magna aquila paruam columbam pugnat}: `el águila grande lucha contra paloma pequeña'	
	
	% \item aquilas puellam paruam amant
\end{frame}


\begin{frame}{Parte 4}

\begin{center}
  \LARGE{La formación de palabras en castellano}
\end{center} 

\end{frame}

\begin{frame}{El análisis de la palabra compleja}

\begin{itemize}
	\item Las \textcolor{blue}{palabras complejas} son aquellas que están integradas por los elementos más pequeños de la lengua que tienen contenido significtivo: los morfemas.
	\item Entre los morfemas, encontramos unas unidades con significado léxico (morfemas léxicos o lexemas) y otras con contenido gramatical (morfemas gramaticales o gramemas).
	\item Hay morfemas que pueden realizarse como palabras independientes (morfemas libres) y otros morfemas que tienen que apoyarse necesariamente a otras unidades (morfemas ligados o afijos).
	\item Los afijos que tienen contenido léxico y que contribuyen a la formación de nuevas palabras se denominan afijos derivativos. 
	\item Los que simplemente transmiten contenido gramatical y no forman nuevas palabras, sino que las flexionan, se denominan afijos flexivos.
\end{itemize}

\end{frame}

\begin{frame}{La segmentación de la palabra compleja}

\begin{itemize}
	\item El dato más importante para identificar un morfema es su recurrencia: el hecho de que el presunto morfema aparezca en otras palabras con un significado semejante. 
	\item En la palabra \it{superrealista} podemos reconocer tres elementos: 
	\begin{itemize}
		\item \it{super-} puede aparecer en otros ejemplos como \it{superhombre, supermasivo}
		\item \it{real-} puede aparecer en otros ejemplos como \it{realidad, realismo}
		\item \it{-ista} puede aparecer en otros ejemplos como \it{comunista, madridista}
	\end{itemize}
\end{itemize}

\end{frame}

\begin{frame}{La segmentación de la palabra compleja}
	
\begin{itemize}
	\item No siempre es fácil descomponer una palabra compleja en sus morfemas constitutivos, especialmente cuando pasamos por sus diferentes estadios de su formación y no identificamos como propia del español alguna unidad.
	\item Ejemplos como \it{minimalismo, minimalista} no se han formado directamente sobre el lexema \it{minim-}, en cuyo caso habrían generado las formas \it{*minimismo, *minimista} sino que provienen de la forma \it{minimal} del inglés.
	\item A veces una palabra puede descomponerse morfológicamente de más de una manera. Para resolver estos casos, debemos fijarnos en otras palabras paralelas o anteder al significado.
\end{itemize}
\end{frame}

\begin{frame}{La segmentación de la palabra compleja}
	
\begin{itemize}
	\item La palabra \it{cafecito} puede descomponerse como \it{caf-ec-ito} dado que \it{-ito} es un morfema fácilmente aislable y sabemos que algunos nombres que toman este se incrementan con un interfijo \it{-ec-} como \it{mes-ec-ito, jefe-ec-ito}.
	
	\item Sin embargo, también puede descomponerse como \it{cafe-c-ito} ya que hay un grupo de palabras que en diminutivo se incrementan con el interfijo \it{-c-} como \it{pastor-c-ito, leon-c-ito}.
	
	\item Otros derivados de \it{café} toman la \it{-e} como parte de la raíz (\it{cafeína}), de manera que, siendo coherentes, parece razonable dar por buena la segmentación \it{cafe-c-ito}.
\end{itemize}
\end{frame}

\begin{frame}{La derivación}

\begin{itemize}
	\item Mediante el procedimiento de la \textcolor{blue}{derivación} formamos nuevas palabras a partir de otras, bien sea añadiendo un afijo, o bien mediante otro procedimiento no afijal.
	\item Es habitual que en el proceso de derivación afijal se eliminen las vocales que sean marca de flexión:
	
	\it{esponj(a) $>$ esponj-os(o) $>$ esponj-os-idad} 
	
	\item En los casos de derivación no afijal conocidos como `formación regresiva', los derivados formalmente más sencillos que la palabra original:
	  
	\it{sostener $>$ sostén, deslizar $>$ desliz, retener $>$ retén} 
\end{itemize}

\end{frame}

\begin{frame}{La derivación}

\begin{itemize}
	\item El procedimiento de derivación más productivo en castellano se efectúa mediante la adición de un afijo pleno.
	\item Sufijación: \it{casa $>$ cas-ero}
	\item Prefijación: \it{coser $>$ re-coser}
	\item Circunfijación o parasíntesis: \it{pobre $>$ em-pobr-ecer}
	\item Se conocen otros dos tipos de afijos que, por sí solos, no derivan palabras: 
	\begin{itemize}
		\item los infijos se colocan dentro del lexema: \it{azuqu-ít-ar}
		\item los interfijos (se colocan entre la base léxica y el sufijo \it{lod-az-al}). 
	\end{itemize}
\end{itemize}

\end{frame}

\begin{frame}{Derivación y flexión: diferencias}

\begin{itemize}
	\item Formalmente, son dos procesos similares en tanto en cuanto consisten en la adición de un afijo a una base léxica.
	\item La derivación forma nuevas palabras, la flexión simplemente modifica la información gramatical de la base léxica.
	\item La flexión en español se realiza solo a través de sufijos.
	\item Prefijos y circunfijos están limitados en español a la derivación.
	\item La derivación puede cambiar la categoría de la base léxica sobre la que se aplica.
\end{itemize}
\end{frame}

\begin{frame}{Estructura de la palabra}

\begin{itemize}
	\item Además de saber aislar los morfemas de los que consta una palabra, el hablante es capaz de reconocer cómo se relacionan entre sí los morfemas de acuerdo con una orden jerárquico.
	\item Los morfemas constitutivos de una palabra como \it{reanudación} puede representarse mediante una estructura de corchetes: 
	 
	 \begin{center}
	 [ [re [a [nud]_{N} a]_{V} ]_{V} ción]_{N}
	 \end{center}
	  
	\item El orden de concatenación de afijos ha sido N $>$ V $>$ V $>$ N.
	\item Algunos derivados de verbos permiten más de un análisis posible: \it{reformulación} puede entenderse como `acción o efecto de reformular' o `nueva formulación.
	
	 \begin{center}
	 fórmula $>$ formular $>$ reformular $>$ reformulación
	 
	 fórmula $>$ formular $>$ formulación $>$ reformulación
	 \end{center}
\end{itemize}

\end{frame}

\begin{frame}{La composición}

\begin{itemize}
	\item La composición es el proceso de formación de nuevas palabras mediante la unión de dos o más lexemas.
	\item Los lexemas que se combinan en las palabras compuestas pueden ser de dos tipos: palabras de la lengua (P) o temas cultos de origen greco-latino (T).
	\begin{itemize}
		\item P + P: \it{hojalata, claroscuro, pelirrojo, vaivén, maniatar}
		\item T + T: \it{logopeda, ecólogo, pediatra, xenófobo, filántropo}
		\item P + T: \it{germanófilo, musicólogo, herbívoro, rumorología}
		\item T + P: \it{ecosistema, geofísico, filocomunista, cardioprotector}
	\end{itemize}
	\item Algunos autores identifican los temas cultos con afijos y clasifican los ejemplos T + P como derivados por prefijación. 
\end{itemize}
\end{frame}

\begin{frame}{Compuestos ortográficos o léxicos}

\begin{itemize}
	\item El compuesto plenamente soldado, llamado \textcolor{blue}{compuesto ortográfico} o \textcolor{blue}{compuesto léxico}, presenta sus componentes unidos gráficamente.
	\item El compuesto tiene un significado propio con un referente único, aunque su significado no es ajeno al de sus constituyentes.
	\item Los constituyentes que incluyen dentro de sí solo pueden ser entidades léxicas, no sintagmas: \it{*guardaloscoches $>$ guardacoches, *paraelsol $>$ parasol}. Hay contraejemplos, como \it{correveidile, hazmerreir} 
\end{itemize}

\end{frame}

\begin{frame}{Compuestos ortográficos o léxicos}

\begin{itemize}
	\item Las marcas morfológicas de género y número se manifiestan solo externamente marcando al compuesto en su totalidad: \it{*aguasardientes $>$ aguardientes, *latinasamericanas $>$ latinoamericanas}. Hay contrajemplos como \it{cualesquiera, quienesquiera}. 
\end{itemize}
    \item Desde el punto de vista fonológico, los compuestos solo tienen un acento primario: \it{álta} + \it{vóz} $>$ \it{altavóz}.
    \item Si tiene más de tres sílaba, es habitual que el compuesto tenga un acento primario y uno secundario: \it{cúmpleáños}.
\end{frame}

\begin{frame}{Sintaxis del compuesto}

\begin{itemize}
	\item Es posible reconocer ciertas relaciones entre los constituyentes de un compuesto. 
	\item Todo compuesto tiene un núcleo que, desde el punto de vista semántico, es un hiperónimo (concepto más general) del compuesto: \it{relieve} es el núcleo de \it{bajorrelieve}.
	\item Cuando el compuesto está dentro de la formación léxica hablamos de \textcolor{blue}{compuesto endocéntrico}: \it{altiplanicie}.
	\item Cuendo el núcleo semántico está fuera del compuesto hablamos de \textcolor{blue}{compuesto exocéntrico}: \it{cienpiés}.
	\item El núcleo es el que impone la categoría gramatical al compuesto y, en la mayoría de los casos, también impone el género.
\end{itemize}

\end{frame}

\begin{frame}{Combinación de categorías léxicas en compuestos}

\begin{itemize}
	\item N + V = V: \it{maniatar}
	\item Adv + V = V: \it{malvivir}
	\item V + N = N: \it{limpiabotas, abrelatas}
	\item V + V = N: \it{duermevela, quitaipón}
	\item N + N = N: \it{hojalata}
	\item N + Adj = N: \it{guardamarina}
	\item Adj + N = N: \it{librecambio, justiprecio}
	\item Adj + Adj = Adj: \it{agridulce, claroscuro}
	\item N + Adj = Adj: \it{pelirrojo}
	\item Adv + Adj = Adj: \it{malsano, bienintencionado}
\end{itemize}
	
\end{frame}

\begin{frame}{Compuestos sintagmáticos}

\begin{itemize}
	\item Los \textcolor{blue}{compuestos sintagmáticos} son determinadas agrupaciones de palabras que se comportan como los compuestos ortográficos, a pesar de que sus componentes se realizan como palabras separadas.
	\item Forman una unidad solidaria y tienen un significado único. 
	\item Existen tres tipos de compuestos sintagmáticos:
	\begin{itemize}
		\item Compuestos preposicionales: \it{ojo de buey, caballo de batalla, piel de gallina, patas de gallo}.
		\item Compuestos yuxtapuestos: \it{perro pastor, pantalón capana, ciudad dormitorio}.
		\item Compuestos de nombre y adjetivo: \it{hilo musical, llave inglesa, buena fe, alta mar}.
	\end{itemize}
\end{itemize}
	

\end{frame}


\end{document}
